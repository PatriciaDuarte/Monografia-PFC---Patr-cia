Contagem de plantas é importante para várias análises agrícolas como estimativa de produtividade, verificação de mortalidade de plantas que pode estar relacionada a doenças ou condições meteorológicas, e inventários que pode servir para várias estimativas como de massa de carbono, informações para planejamento de irrigação, entre outras.\cite{Daliakopoulos2009} 

Nesta pesquisa foram propostas duas metodologias para a contagem de plantas, sendo que elas têm uma fase de processamento comum (conversão das imagens para níveis de cinza, binarização, tratamento morfológico, e aplicação da transformada da distância). Em seguida são experimentados dois métodos para a detecção das plantas (segmentação \textit{watershed} e detecção de picos através da filtragem máxima local). Posteriormente é feita uma rotulação da plantas identificadas. 

Ao aplicar as metodologias para imagens de satélite coletadas pela API Google Maps, o método que usa a filtragem máxima local obtém resultados mais precisos, com 92.03\% de precisão para a contagem de jabuticabeiras e 92.88\% para a contagem de coqueiros, versus 83.94\% e 87.91\%, respectivamente, obtidos pela metodologia que usa segmentação \textit{watershed}. 

É importante destacar, que esta forma de análise dos resultados é propensa a erros, pois uma estimativa satisfatória do número de plantas não significa necessariamente que as plantas foram detectadas corretamente. Assim, na disciplina de PFC-2 será aprimorada aplicando métodos de avaliação da detecção de plantas mais robustos, que se baseia em critério como de falsos positivos, falsos negativos e verdadeiros positivos. Contudo, para a aplicação dessas medidas é necessário identificar manualmente cada planta na imagem, que será etapa feita futuramente. Além disso, será feito experimentos de índices de vegetação para a detecção dos pixels de vegetação, e usar informações geométricas, como o espaçamento entre plantas para eliminar falsos positivos.