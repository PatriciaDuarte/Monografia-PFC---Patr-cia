\section{Considerações iniciais}


Este capítulo explora metodologias existentes na literatura que são capazes de realizar detecção e contagem de plantas. Primeiramente será descrito o tipo de imagem pesquisada, em seguida a espécie da planta, as técnicas usadas e a sua taxa de precisão.


%Nome autores: Vibha L, P Deepa Shenoy, Venugopal K R, L M Patnaik
%Nome do artigo:Robust Technique for Segmentation and Counting
%of Trees from Remotely Sensed Data
Segundo \citeonline{Vibha2009}, o tipo de imagens coletadas e estudadas por eles, foram imagens de sensoriamento remoto do satélite QuickBird. O objetivo deste trabalho, foi criar uma abordagem para a segmentação da imagem e contar árvores. Segundo os autores, fazer contagem manualmente em terrenos florestais é uma tarefa maçante, e com isso, teve motivação de desenvolver este projeto para potencializar o processo de contagem. Por suas imagens estarem com uma qualidade muito ruim, foi aplicada a técnica de filtragem média no pré-processamento. Na etapa de contagem de árvores teve duas fases: Na primeira, faz o aprimoramento usando limiarização automática e na segunda fase,foi criado um modelo para a árvore chamado matriz de imagem, juntamente com uma função de mapeamento. Eles fizeram seus programas em Matlab 7 e a acurácia média da contagem foi de 88\%. 


%Nome autores: Attilio Antonio Disperati, João Roberto dos Santos, Paulo Costa de Oliveira Filho e Till Neeff 
%Nome titulo: Aplicação da técnica “filtragem de locais máximas” em fotografia aérea digital para a contagem de copas em reflorestamento de Pinus elliottii
\citeonline{Disperati2007}, o tipo de imagens coletadas para serem estudadas, foram imagens de sensoriamento remoto. O objetivo deles era contar o topo das árvores de Pinus elliottii. A área das plantações das imagens que foram desenvolvidas, está na Floresta Nacional de Irati que situa-se no Paraná. Aplicaram a técnica de filtragem local máxima. O principal software usado por eles foi feito em C++. Ao usarem imagens de 600 dpi, tiveram 97\% de acurácia na identificação das copas das árvores. Nas imagens com 100, 200 e 300 dpi foram aplicados quatro tamanhos de filtros combinados com a estatura das copas das imagens. Na pesquisa dos autores, foi comprovado que quanto menor a estatura do filtro, maior foi o número de pontos de máximas encontrados. Ao usar imagem com resolução de 300 dpi, tiveram 70,7\% de acurácia.


%Titulo: Tree Crown Detection, Delineation and Counting in UAV Remote Sensed Images: A Neural Network Based Spectral–Spatial Method
%Autores: Ramesh Kestur , Akanksha Angural, Bazila Bashir, S. N. Omkar , M. B. Meenavathi , Gautham Anand 
Segundo \citeonline{Kestur2018}, as imagens estudadas por eles foram feitas por meio de dois UAVs de características distintas, porém os dois produzem imagens de alta resolução espacial. O objetivo deste trabalho, foi classificar a espectro-espacial dessas imagens RGB de alta resolução espacial para detectar e contar árvores. O tipo de plantas estudadas por eles foram bananeiras, mangueiras e conqueiros. Foi feito classificação supervisionada usando ELM(aprendizado de máquina extremo). O ELM foi modelado para valores RGB como vetores de recursos de entrada e classe de saída binária. Posteriormente, foi aplicado classificação espacial usando técnica de filtragem de propriedades geométricas com limiares.  Fazeram segmentação da imagem usando Watershed aplicando o método de Transformação e Distância. Por fim, comparam o método ELM com o método espectro-espacial do k-means. Os resultados mostraram que o método ELM teve melhor desempenho que o K-means.


%Improving the Precision of Tree Counting by Combining Tree Detection with crown Delineation and classification on Homogeneity Guided Smoothed High Resolution(50 cm) Multispectral airbone Digital Data
Segundo \citeonline{Katoh2012} usaram imagens de  dados de sensoriamento remoto. O local estudado se localiza em uma floresta do campus da Universidade de Shinshu-Japão. Fizeram estudo de coníferas de várias espécies, tais como: C.obtusa, C.pisifera, P. densiflora, L.kaempferi, C. japonica e árvores com as folhas de Carvalho. O foco deste trabalho foi detectar as espécies e fazer a contagem delas. O processo seguido no trabalho deles foi: 1- separar as regiões do tipo cobertura homogênea(separadas por textura ou níveis de cinza); 2- Aplicou-se a técnica de filtragem máxima local; 3- Thresholding; 4- Usou-se classificação supervisionada com uma regra de decisão de máxima verossimilhança, para comparar as espécies de árvores. Por fim, a precisão do número total de árvores contadas por espécie, foi superior a 84\%.


%Titulo: Oil Palm Tree Detection with High Resolution Multi-Spectral Satellite Imagery
%Autores: Panu Srestasathiern  and Preesan Rakwatin
No trabalho de \citeonline{Sresta2014}, foi estudado imagens de satélites de alta resolução espacial, e o foco foi contar plantas de dendezeiros. O processo feito para fazer a contagem é começar aplicando o índice de vegetação, já para melhorar a separabilidade entre as copas dos dendezeiros e o fundo da imagem, foi aplicado a transformação de classificação. Posteriormente, foi aplicado o algoritmo de supressão não máximo, e por fim, foi aplicada a análise semi-variograma. Neste trabalho, a contagem teve 90\% de taxa de detecção em comparação com a abordagem manual.


%Titulo: Deep Learning Based Oil Palm Tree Detection and Counting for High-Resolution Remote Sensing Images
%Autores: Weijia Li ; Haohuan Fu; Le  and Arthur Cracknell 
Segundo \citeonline{Li2017}, foram estudadas imagens de sensoriamento remoto de alta resolução em seu trabalho. O foco foi detectar e contar dendezeiros. As características das árvores estudadas por eles são lotadas e tem sobreposições. O processo que eles seguiram, foi primeiramente treinar uma rede neural convolucional, fez ajustes nela e posteriormente, preveram os rótulos de todas as bases de imagens que são coletadas pela técnica da janela deslizante. Usaram o framework Tensorflow e ainda compararam com 3 outros métodos. A metodologia criada por eles teve o melhor resultado, obtendo 96\% de árvores detectadas corretamente, em comparação com a contagem manual.


%Titulo: Comparing Boosted Cascades to Deep Learning Architectures for Fast and Robust Coconut Tree Detection in Aerial Images
%Autores: Steven Puttemans , Kristof Van Beeck  and Toon Goedemé
Segundo \citeonline{Puttemans2018}, as imagens usadas por eles são através de imagens aéreas e a planta específica que foi explorada são coqueiros. O objetivo deles foi fazer deteção e classificação, usando dois tipos diferentes de abordagens e depois compará-las. As técnicas que os autores estudaram foi a aplicação de aprendizado profundo e cascatas(técnicas que foram exploradas em seus respectivos trabalhos correlatos). Usaram 3 estruturas disponíveis para o treinamento: 1- Cascata aumentada usando OPENCV; 2- Cascata reforçada usando MATLAB. Para desenvolver o modelo de deep learning, usaram C e CUDA. Usaram o framework Darknet para classificação e o YOLOv2 para detecção. A taxa de precisão da técnica da cascata melhor impulsionada foi a média de 94,56\%, enquanto o modelo de aprendizado profundo tem 97,4\%.


%Detection of Individual Tree Crowns in Airborne Lidar Data – Koch 2006
%(Detecção de copas de árvores individuais em dados aéreos de Lidar )
Segundo \citeonline{koch2006}, as imagens estudadas em seu trabalho foram capturadas através da base de dados do Lidar aéreo. O foco deste artigo foi efetuar detecção de copas e altura das árvores, e o tipo delas é árvores coníferas. O estudo delas foi feito para saber se a estrutura delas é adequada,foi mencionado que essas informações são importantes na área de Engenharia Florestal. As implementações foram feitas na linguagem C++. Eles usaram suavização gaussiana para separar a diferença do tamanho das árvores. Já para detectar a coroa das árvores, foi usado o filtro máximo local. Por fim, 87,3\% das árvores são detectadas corretamente ou satisfatórias.


%Comparison of airborne and satellite high spatial resolution data for the identification of individual trees with local maxima filtering – Wulder 2004
%(Comparação de dados de alta resolução espacial no ar e por satélite para a identificação de árvores individuais com filtragem máxima local)
Segundo \citeonline{wulder2004}, as imagens estudadas foram da base de imagens de satelite IKONOS, por elas terem uma boa resolução.  Os tipos de árvores estudados foram Abeto-de-Douglas e Centro Vermelho Ocidental, que são plantas típicas de florestas.  Os autores também fazem comparações de imagens entre os satélites MEISII e IKONOS. A técnica usada para a detecção das árvores foi a de filtragem máxima local. O IKONOS teve 85\% de precisão e 51\% de erros de comissão, já o MEISII tem precisão de 67\% e 22\% de erro de comissão.


%Individual tree detection on variable and fixed Window size local maximum filtering applied to IKONOS imagery for enven-aged Eucalyptus plantation forests – GEBRESLASIE 2011
Segundo o autor \citeonline{Gebreslasie2011}, o  tipo de imagem estudado por ele, é por via de imagens de sensoriamento remoto por satélite. O tipo de árvore estudado por eles é plantações de Eucalypto de mesmo tamanho. As técnicas usadas por eles foi suavização gaussiana e classificação de quebra natural para determinar o limiar para a remoção de pixels de áreas extremamente brilhantes e escuras nas imagens. Foi aplicado também a técnica de semivariograma para determinar tamanhos de janela variáveis para filtragem de máximos locais dentro de um plantio. Um tamanho fixo de janela para filtragem máxima local também foi aplicado usando espaçamento pré-determinado. A precisão geral usando um tamanho de janela variável foi de 85\%, enquanto um tamanho fixo de janela resultou em uma precisão de 80\%.


%Automated extraction of tree and plot-based parameters in Citrus orchards from aerial images
%(Extração automatizada de parâmetros de árvores e parcelas em pomares de citros a partir de imagens aéreas) – Recio 2013
Segundo \citeonline{recio2013}, o tipo de imagens estudadas são a partir de imagens aéreas de alta resolução espacial, a área de estudo se localiza em Valência-Espanha. O objetivo foi extrair fração da cobertura arbórea, número de árvores e padrões de plantio. O tipo de planta estudado é Citros. O processo de desenvolvimento deles possui as seguintes etapas: Aplicação de uma classificação não-superviosionada com o algoritmo k-means, seguida pela identificação automática das classes que representam as árvores. Após este procedimento, a árvore é individualizada usando morfologia na imagem binarizada. Foi feita a classificação de 3 tipos de árvores para o cálculo de acurácia: Independentes, alinhadas e jovens. As árvores independentes tem valores mais altos, com o percentual de  95,96\%. As árvores alinhadas teve o percentual de 83,78\%, já as árvores jovens tiveram 87,26\% de acurácia.


%A tree counting algorithm for precision agriculture tasks
%(Um algoritmo de contagem de árvores para tarefas de agricultura de precisão) – Santoro 2013
Segundo o autor \citeonline{franco2013}, os tipos de imagens usados foram a partir de dados do sensor GeoEye-1.  As árvores estudadas são Citrus. Para a construção do algoritmo, o mesmo possui 4 etapas: Filtro de suavização assimétrico; filtro mínimo local; camada de máscara e operador de agregação espacial. O objetivo deste trabalho foi contar as árvores. Nele descreve as dificuldades obtidas. O algoritmo proposto foi avaliado usando as seguintes métricas de objetos: Porcentagem de detecção(DP) e fator de ramificação(BF).
\begin{equation}
    DP = \frac{100 \ X \ TP}{TP + TN}  
\end{equation}
\begin{equation}
    BF = \frac{100 \ X \ FP}{TP + FP}
\end{equation}
onde,

TP = Verdadeiro positivo ou árvores corretamente identificadas pelo usuário e através do procedimento automático;

FP = Falso positivo ou árvores identificadas pela abordagem automática, mas não pela do utilizador;

TN = Verdadeiro negativo ou árvores identificadas pelo usuário, mas não através do procedimento automático.

A precisão planimétrica dos centróides de árvores decorrentes do filtro de suavização assimétrico proposto também foi avaliada pelo Root Mean Square Error (RMSE), sem incluir árvores erroneamente identificadas (erros de comissão). Essa avaliação foi feita assumindo que as medidas obtidas pelo usuário eram precisas e considerando-as como referência no cálculo do RMSE para avaliação à distância:
\begin{equation}
    RMSE = \sqrt{\frac{\sum_{i=1}^{n}(D_{m,i} - D_{r,i})^{2}}{n}}
\end{equation}
onde,

n =  número de observações de referência;

$D_{r,i}$ = observações obtidas pelo usuário;

$D_{m,i}$ = observações obtidas através do procedimento.


%Applying Image Analysis and Probabilistic Techniques for Counting Olive Trees in High-Resolution Satellite Images
%(Aplicação de análise de imagem e técnicas probabilísticas para contagem de oliveiras em imagens de satélite de alta resolução) - González 2007
Segundo o autor \citeonline{Gonzalez2007}, as imagem usadas usadas para o estudo, são a partir de imagens de satélite de alta resolução. O local de estudo é localizado no sul da Espanha, o objetivo foi fazer contagem de Oliveiras. A metodologia usada para fazer a contagem de Oliveiras, é primeiramente pegar as características e o tamanho semelhante das árvores, em seguida produzir uma medida probabilística que já detecta qual é uma Oliveira. As técnicas usadas, é fazer a localização dos centróides, posteriormente, usaram escala de cinzas que forneceu melhores resultados.  Usaram a linguagem c++ e biblioteca OpenCV. Por fim, para testar a adequação do método proposto, foram comparados os resultados do programa com o número de árvores visualmente contadas por um operador de ortofotografias aéreas coloridas. Nesta comparação, foram diferenciados falso-positivos (PF) e negativos (FN). Um candidato é definido como FP se for detectado erroneamente como uma oliveira e FN se for detectado erroneamente como uma não-oliveira.

%DETECTION AND COUNTING OF ORCHARD TREES FROM VHR IMAGES USING A
%GEOMETRICAL-OPTICAL MODEL AND MARKED TEMPLATE MATCHING – MAILLARD 2016        
Segundo \citeonline{Maillard2016}, as imagens estudadas por eles são de alta resolução. As mesmas foram extraídas de diversos lugares do mundo, a partir do aplicativo web Google Earth. O objetivo deles é detectar e contar árvores. Usaram a abordagem correspondência de modelo óptico geométrico(GOM). O algoritmo feito neste trabalho, contém os seguintes estágios: 1. O usuário insere a porcentagem de sobreposição permitida; 2. Se o usuário não inserir a entrada, o próprio programa calcula a semelhança entre a amostra e todas as possibilidades de parâmetros de iluminação; 3. Foi calculado um valor de similaridade para cada pixel; 4. Se a similaridade for multiespectral, repita o estágio 3 para todas as bandas e tornar a similaridade cumulativa; 5.Classifique os pixels diminuindo a semelhança e armazene as coordenadas(O valor tem que ser maior que o mínimo permitido); 6. Coloque uma marca de árvore temporária(círculo) no próximo local de pixel com o maior valor de similaridade; 7. Verifique se o espaço já está ocupado por uma árvore. Se alguma sobreposição for permitida, verifique se o número de pixels diferentes de zero é menor que a porcentagem de sobreposição permitido; 8. Validação dos resultados. Por fim, como resultado da contagem das árvores de laranjas, nozes e manga tiveram acima de 90\% de precisão, já as macieiras tiveram abaixo de 75\% de precisão. 

%Classification of hazelnut by self-organizing maps
%(Classificação de pomares de avelã por mapas auto-organizados) – Tasdemir 2010
No trabalho de \citeonline{tasdemir2010}, as imagens usadas para desenvolver a pesquisa são a partir de imagens de sensoriamento remoto. A árvore estudada é Avelãs. A abordagem que os autores usaram, foi mesclar as informações espectrais e espaciais. Proporam um mapa auto-organizado que pega essas informações sem cálculo adicional de textura. A metodologia que eles usaram é encontrar a cobertura do solo a partir do método baseado em pixels, através de aprendizado de máquina. Primeiro aplicaram a abordagem de aprendizado de quantização de vetores, consideraram cada pixel junto com seus vizinhos em uma janela de tamanho predeterminado e usaram todos os valores espectrais na janela como vetor de característica. 

%Automatic Detection and Segmentation of Orchards Using Very High Resolution Imagery-Aksoy2012
%(Detecção automática e segmentação de pomares usando imagens de alta resolução)
Segundo \citeonline{Aksoy2012}, as imagens usadas neste trabalho foram extraídas no aplicativo web google earth e de satélites de alta resolução. Os autores proporam um algoritmo não-supervisionado para detectar e segmentar espécies das plantas, que são Avelãs e árvores cítricas. Primeiramente, foi feito o aprimoramento de possíveis localizações de árvores usando filtros isotrópicos de granularidade. Posteriormente, a regularidade dos padrões de plantio foi quantificada usando perfis de projeção das respostas do filtro em várias orientações. O resultado foi pontuação de regularidade em cada pixel para cada granularidade e orientação. Já na etapa de segmentação, foi mesclada iterativamente pixels e regiões vizinhos pertencentes a padrões de plantio semelhantes de acordo com as semelhanças de suas pontuações de regularidade e obteve os limites de pomares individuais juntamente com estimativas de suas granularidades e orientações.

%An Automatic Method for Counting Olive Trees in Very High Spatial Remote Sensing Images - Bazi2009
No artigo de \citeonline{Bazi2009} foi usada imagens de sensoriamento remoto de alta resolução espacial. O objetivo dos autores foi efetuar contagem de Oliveiras. O processo que os autores fizeram para fazer a contagem, foi primeiramente aplicar morfologia matemática, em seguida separaram as classes do solo e copas através do classificador gaussiano(HPC), em seguida, o blob que representa as Oliveiras, foram contados de forma automática. Como resultado,  o algoritmo contou 1124 de de 1167 Oliveiras. Segundo os autores, foi um resultado promissor.

%An Automatic Approach for Palm Tree Counting in UAV Images - Bazi2014
Segundo \citeonline{Bazi2014}, o objetivo deles foi fazer contagem de Palmeiras em imagens UAV. Primeiramente, foi feito o treinamento com o SIFT(Scale Invariant Feature Transform) para extrair conjuntos de pontos-chave. Posteriormente, esses pontos-chaves foram analisados com o classificador Extreme Learning Machine (ELM), que também foi treinado no conjunto de pontos-chave de palma e sem palma. Como saída, o classificador ELM marcou cada palmeira detectada por vários pontos-chave. Em seguida, para capturar a forma de cada árvore, foi proposto mesclar esses pontos-chave com um método de contorno ativo baseado em conjuntos de níveis(LS). Por fim, a textura das regiões obtidas por LS com padrões binários locais(LBPs) foi analisada, para distinguir palmeiras de outras vegetações.[VOLTAR MECHER AQUI!!!!]

%Oil Palm Counting and Age Estimation from WorldView-3 Imagery and LiDAR Data Using an Integrated OBIA Height Model and Regression Analysis - Rizeei2018

No trabalho de \citeonline{Rizeei2018}, foram estudadas imagens a partir do satélite Worldview-3 e de imagens aéreas de detecção e alcance da luz(LiDAR) no ar. Os autores estimaram a idade e contaram Dedenzeiros. Primeiramente, aplicaram o algoritmo de máquina de vetor de suporte(SVM) de análise de imagem baseada em objeto para efetuar a contagem de dendezeiros. Foi feita análise de sensibilidade em quatro tipos de kernel SVM com parâmetros de segmentação associados para obter o melhor delineamento da cobertura da coroa. A extração da copa da árvore foi integrado ao modelo de altura e métodos de multiregressão para estimar com precisão a idade das árvores. O modelo de multiregressão com tamanhos de vários núcleos foi examinado para obter o modelo mais otimizado para estimativa de idade. Os modelos aplicados foram treinados e examinados em cinco diferentes plantações de dendezeiros. A estimativa geral da contagem de dendezeiros, teve uma precisão geral de 98.80\%, já para a precisão geral da estimativa de idade foi de 84.91\%. 

%An yield estimation in citrus orchards via fruit detection and counting using image processing - Dorj2017
Segundo \citeonline{Dorj2017}, as imagens estudadas por eles foram extraídas através de câmera fotográfica. O foco deste trabalho foi detectar e contar citros na árvore.O algoritmo de contagem de citros consistiu nas seguintes etapas: 1. Converter imagem RGB em HSV; 2. Fazer limiarização; 3. Fazer detecção da cor laranja; 4. Fazer remoção de ruído; 5. Fazer segmentação pelo algoritmo de bacias hidrográficas; 6.Fazer contagem. O algoritmo foi comparado com a abordagem manual, e o valor da análise de regressão foi de $R^{2}$= 0,93.

%Tree Crown Detection on Multispectral VHR Satellite Imagery - Daliakopoulos2009
No trabalho de \citeonline{Daliakopoulos2009},foram estudadas imagens de satélite multiespectrais de resolução muito alta(VHR). O foco dele foi detectar árvores por meio do seu tamanho, ao invés de espécies específicas.  O método usado, possui combinação dos limites da faixa Vermelha e do Índice de Vegetação de Diferenças Normalizadas(NDVI) e o método de detecção de blob do Laplaciano do Gaussiano(LOG). 

%Utilização de imagens de sensoriamento remoto de alta resolução para realizar a contagem de copas em povoamento de Eucalyptus spp.
No trabalho de \citeonline{Reis2007}, foram usadas imagens de sensoriamento de alta resolução para fazer contagem de Eucalyptus ssp. Primeiramente, foi aplicado o filtro de Lee e em seguida o classificador não-supervisionado ISODATA e foi feita a contagem com o sistema SIG(Sistemas de Informações Geográficas). A abordagem usada por eles obteve acertos de 93,58\%.

%A graph-based segmentation algorithm for tree crown extraction using airbone LIDAR data – STRIMBU 2015
No trabalho de \citeonline{strimbu2015}, as imagens estudadas foram extraídas a partir do LIDAR aéreo. A floresta que  tos autores estudaram é predominado por pinheiros com as seguintes espécies:  Pinus taeda L. (PT), Pinus palustris Mill. (PP), Quercus falcata Michx. (QF), Quercus alba L. (QA)  Liquidambar
styraciflua L. (LS).  As técnicas aplicadas neste trabalho são as seguintes:%Não sei detalhar as técnicas, vou jogar este artigo no lixo.
