% Comando simples para exibir comandos Latex no texto
%\newcommand{\comando}[1]{\textbf{$\backslash$#1}}

\section{Considerações iniciais}

Segundo \cite{Gonzales2010}, a área de processamento digital de imagens, que se refere ao processamento de imagens por um computador digital, está em expansão e possui aplicações em duas amplas categorias: (1) o aprimoramento de informações pictóricas para interpretação humana; e (2) a análise automática por computador para extrair informações de uma cena. 

Historicamente, a área de processamento digital de imagens começou a ser usada na década de 20 no século passado. Inicialmente ela era usada para melhorar a qualidade de transmissões de televisão, e com o passar do tempo, as técnicas foram aprimoradas para essa aplicação. Três décadas mais tarde, a área voltou a se expandir rapidamente, por decorrência do lançamento de computadores digitais de grande porte para uso em programas espaciais. Nesse período, usavam-se diversas técnicas para o realce e a restauração de imagens de programas espaciais, como nas expedições tripuladas da série Apollo, por exemplo. Essas duas tecnologias, realce/restauração de imagens e computadores digitais, andam juntas atualmente, e formam a base para a área de processamento digital de imagens.


Conforme \cite{Gonzales2010}, não há um consenso entre autores sobre onde termina o processamento de imagens e onde começam áreas relacionadas, tais como análise de imagens e visão computacional. Algumas vezes, a distinção é feita por definir processamento de imagens como a área na qual, ambas a entrada e a saída do processo são imagens. A compreensão do conteúdo de imagens, é denominada de análise de imagens ou de visão computacional, dependendo do nível de complexidade contido. Análise de imagens, normalmente é usada para se referir a extração de informações predefinidas a partir de uma imagem, tal como detecção de um objeto, classificação, contagem, entre outras. A visão computacional, é usada para referir a processos mais complexos, tal como o sistema artificial de visão de um robô ou de um veículo autônomo. Contudo, muitos autores não faz a distinção entre processamento de imagens, análise de imagens e visão computacional. Neste trabalho, como usaremos técnicas clássicas de realce e de análise de imagens, também não focaremos nesta distinção, usando assim o termo processamento digital de imagens ou simplesmente processamento de imagens.

O processamento digital de imagens pode ser aplicado em áreas como medicina, para auxílio ao diagnóstico médico através de imagens, na arqueologia para fazer restauração de imagens, na biologia para a análise de imagens microscópicas, na área de agricultura para detecção de doenças em plantações, para fazer estimativas de produtividade agrícola e análise de vigor vegetativo, na segurança pública para controle de acesso, entre várias outras áreas. Este trabalho, foca na detecção e contagem de árvores, o que tem se tornado uma possibilidade real de automação computacional, devido ao surgimento de satélites de alta resolução, de veículos aéreos de custo acessível usados para o propósito de monitoramento, de computadores de alto poder de processamento capaz de analisar imagens de satélites de alta resolução em segundos, e de variadas técnicas computacionais que podem ser empregadas para a contagem de árvores. Na seção a seguir, apresentamos a motivação para esta pesquisa, tanto do ponto de vista dos benefícios da aplicação, quanto do ponto de vista científico.
 

\section{Motivação}

Detecção e contagem de árvores é importante para gerenciamento agrícola e de florestas, provendo informações precisas para planejamento de irrigação, aplicação de fertilizantes, estimativa de produtividade, inventários de quantidade de biomassa e de estoque de carbono. Um acompanhamento temporal de contagem de árvores também pode prover informação para fiscalização e para verificação de mortalidade de árvores, que pode ser um índice importante quando associado à doenças, pragas, condições meteorológicas e fatores do ecossistema~\cite{Daliakopoulos2009}.

O trabalho de contagem de árvores é tradicionalmente feito por pesquisas de campo que são demoradas, têm alto custo e são altamente suscetíveis a erros. Contudo os avanços dos últimos anos na qualidade de imagens de satélite e o surgimento de veículos aéreos não-tripulados (VANTs) de custo acessível têm aberto caminho para várias pesquisas de sensoriamento remoto com variados propósitos. Satélites comerciais como Quickbird, Orbview e Ikonos produzem imagens de alta resolução permitindo a detecção, identificação e contagem de objetos na superfície do solo com alta precisão~\cite{Vibha2009, Srestasathiern2014, Li2017, recio2013, franco2013, Gonzalez2007}. VANTs também têm sido amplamente usados, e suas principais vantagens em relação a satélites são a resolução de imagens e possibilidade de obter imagens a qualquer dia e horário~\cite{Disperati2007,Kestur2018}. Satélites são críticos em determinadas épocas do ano devido a presença de nuvens. 


Dada a disponibilidade de imagens de sensoriamento remoto, pesquisadores têm concentrado no desenvolvimento de técnicas que possam vir a substituir a análise humana das imagens captadas. Existe uma variedade de abordagens computacionais para a detecção de árvores, tais como casamento de \textit{template}, segmentação por crescimento de regiões, detecção de picos e métodos de aprendizado profundo. Métodos de casamento de \textit{template} constroem uma série de modelos para caracterizar os aspectos de árvores, levando em consideração a geometria da copa e propriedades radiométricas. Uma vez construídos os \textit{templates} um procedimento de janela deslizante é implementado para buscar os melhores casamentos, isto é, os locais de maiores probabilidades de existência de árvores. Métodos de segmentação por crescimento de regiões partem de vários pixels semente e vai agregando novos pixels na vizinhança conforme algum critério de similaridade. Métodos de filtragem máxima local são usados para detectar picos locais, dados por uma alta saturação de verde normalmente encontrada no interior das copas das árvores. Tal procedimento é aplicado por mecanismo de convolução usando uma janela deslizante de um tamanho específico dada pela discretização de uma função gaussiana. Métodos de aprendizagem profunda usando a abordagem convolucional têm sido empregados com sucesso em várias tarefas de visão computacional. Estes métodos aprendem uma extração de características através de uma representação interna de pesos da rede neural, que melhor classificam os dados. No Capítulo~\ref{chapter:correlatos} serão apresentadas, as principais pesquisas de cada uma das categoria citadas, para a detecção e contagem de árvores.


\section{Formulação do problema de pesquisa}

Dado uma imagem de satélite colorida (RGB) de alta resolução (de ao menos 2m/pixel) de uma área agrícola delimitada, nosso problema consiste em detectar e contar a quantidade de árvores utilizando técnicas automáticas de processamento de imagens. Outras possíveis informações que podem ser usadas para a resolução do problema, inclui o espaçamento entre árvores e estimativa da área ocupada pela copa de cada árvore. O resultado gerado pela resolução do problema incluirá a localização de cada árvore e/ou a delimitação da área ocupada por sua copa e a contagem das árvores. Pretende-se avaliar os resultados gerados gerados ao final deste trabalho com base em métricas tradicionais de detecção de objetos, tal como precisão, sensitividade, especificidade, entre outras  que são apresentadas no Capítulo~\ref{chapter:desenvolvimento}, em comparação com imagens de \textit{ground-truth} geradas por anotação manual.


\section{Objetivo}

Nesta pesquisa, técnicas de processamento digital de imagens serão aplicadas na área de agricultura, visando compor uma abordagem efetiva para detecção e contagem de árvores com base em imagens de satélite. Para se chegar a abordagem, pretende-se comparar técnicas alternativas para as subtarefas do processo, avaliando estas, através de métricas para mensurar a qualidade dos resultados obtidos.


\section{Metodologia}
As imagens serão coletadas pela API Python do Google Maps. Será aplicado o processamento que consiste de índices de vegetação, binarização, tratamento morfológico e aplicação da transformada da distância. Posteriormente, serão comparados  métodos clássicos de detecção de objetos. Até o momento experimentamos \textit{Watershed} e da filtragem máxima local. Por fim, será feita a rotulação dos objetos identificados, onde o número de rótulos serão o número de árvores estimadas.


\section{Principais resultados e contribuições iniciais}

Até o momento foram desenvolvidas duas metodologias para a contagens de árvores a partir de imagens de satélite de alta resolução. As metodologias consistem de um processamento comum à ambas: conversão das imagens para níveis de cinza, binarização, tratamento morfológico e aplicação da transformada da distância. Na segunda etapa do processo, comparamos segmentação por inundação (\textit{watershed}) com a abordagem de detecção de picos através do método de filtragem máxima local. Na terceira fase, é feita a rotulação dos objetos identificados, onde o número de rótulos corresponderá aos número de árvores estimado. Para PFC2 pretende-se comparar o desempenho de índices de vegetação, ajustar adequadamente os parâmetros envolvidos em cada metodologia, desenvolver uma base de \textit{ground-truth} e validar as metodologias com métricas objetivas de detecção de objetos, tais como sensitividade, especificidade, entre outras. 


\section{Obstáculos esperados e desafios iniciais}

%Conforme destacado na literatura, a contagem de árvores com base em imagens aéreas, coletadas por satélites ou veículos aéreos não-tripulados (VANTs), é importante sob vários aspectos. Por exemplo, \citeonline{Daliakopoulos2009} e \citeonline{Dorj2017} utilizaram a contagem de árvores com base em imagens para estimar o rendimento de culturas.
 
%\citeonline{Daliakopoulos2009} desenvolveu uma metodologia baseada em visão computacional, para monitorar o número de árvores com o passar do tempo. Os autores argumentam que essas contagens ao longo do tempo, pode resultar em índices importantes quando associados a doenças de árvores, condições climáticas e comportamento do ecossistema.

%Segundo \citeonline{Reis2007}, foi resolvido usar abordagens para contagem de árvores para resolver o problema de inventários florestais. Segundo os autores, os dados de inventários florestais têm sido coletados principalmente por pesquisas de campo, que são dispendiosas e demoradas, e muita das vezes com erros devido a falta de treinamento da equipe de campo.

Conforme reportado na literatura, apesar dos variados benefícios de análises, existem vários desafios para o desenvolvimento de métodos altamente precisos para a detecção e contagem de árvores. Entre estes desafios, estão a presença de nuvens que pode resultar em oclusões nas imagens capturadas, problemas envolvendo resolução de imagem e a dimensão da copa das árvores \cite{Srestasathiern2014}, além de problemas de sobreposição da copa das árvores \cite{Disperati2007}.

Segundo \cite{Daliman2016}, a análise de árvores individuais com base em imagens de sensoriamento remoto é um problema complexo, pois há variações do tamanho, da forma e da resposta espectral da copa das árvores. Com isso, o que é detectado como objetos únicos podem de fato corresponder à galhos da uma mesma árvore, ou a um grupo de árvores.

Outros desafios para o desenvolvimento de métodos de contagem de árvores, de acordo com o estudo de \cite{Daliman2016}, é que poderá haver erros de detecções de árvores, e isso ocorre por causa da proximidade de árvores vizinhas, árvores que encobrem outras árvores, árvores à sombra ou árvores que tem baixo contraste espectral com o fundo. Apesar desses problemas, é de extrema importância fazer a detecção devido as análises agrícolas e aplicações subsequentes.

No resultados iniciais pode-se perceber os problemas citados por \cite{Daliman2016}, principalmente devido ao fato de sobreposição da copa das árvores. Pretendemos mensurar quantitativamente e propor soluções para amenizar este problema na disciplina de PFC2 este será desenvolvidas imagens que \textit{ground-truth} que permitirá mensurar adequadamente a qualidade dos resultados obtidos.

\section{Organização da monografia}

O restante desse trabalho está dividido da seguinte forma. No Capítulo~\ref{chapter:conceitosPDI} são apresentados os principais conceitos básicos de processamento de imagens necessários para a compreensão das metodologias em desenvolvimento para a contagem de árvores. No Capítulo~\ref{chapter:correlatos} as pesquisas correlatas de contagens de plantas são categorizadas de acordo com as técnicas aplicadas e discutidas. No Capítulo~\ref{chapter:desenvolvimento} são apresentados o desenvolvimento e os resultados iniciais. Por fim, o Capítulo~\ref{chapter:conclusao} apresenta conclusões sobre o desenvolvimento feito na disciplina de PFC1 e o cronograma. 
