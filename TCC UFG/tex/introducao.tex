% Comando simples para exibir comandos Latex no texto
%\newcommand{\comando}[1]{\textbf{$\backslash$#1}}

\section{Considerações iniciais}

Segundo \citeonline{Gonzales2010}, a área de processamento digital de imagens, que se refere ao processamento de imagens por um computador digital, está em grande crescimento e possui aplicações em duas amplas categorias: (1) o aprimoramento de informações pictóricas para interpretação humana; e (2) a análise automática por computador para extrair informações de uma cena. 

Historicamente, a área de processamento digital de imagens começou a ser usada na década de 20 no século passado, e inicialmente era usada para melhorar a qualidade de transmissões de televisão, e com o passar do tempo, as técnicas foram aprimoradas para essa aplicação. Três décadas mais tarde, a área voltou a se expandir rapidamente, por decorrência dos lançamentos de computadores digitais de grande porte para uso em programas espaciais. Nesse período, usavam-se diversas técnicas para servir de base para métodos aprimorados de realce e restauração de imagens de programas espaciais, como as expedições tripuladas da série Apollo, por exemplo. Essas duas tecnologias (realce/restauração de imagens e computadores digitais) andam juntas atualmente, e formam a base para a área de processamento digital de imagens, que é rica por possuir muitas aplicações em áreas multidisciplinares.


Conforme \citeonline{Gonzales2010}, não há um consenso entre autores sobre onde termina o processamento de imagens e onde começam áreas relacionadas, tais como análise de imagens e visão computacional. Algumas vezes, a distinção é feita por definir processamento de imagens como a área na qual ambas a entrada e a saída do processo são imagens. A compreensão do conteúdo de imagens é denominada de análise de imagens ou de visão computacional, dependendo do nível de complexidade implícito na afirmação “compreensão do conteúdo”. Análise de imagens normalmente é usada para se referir a extração de informações predefinas a partir de uma imagem, tal como detecção de um objeto, classificação, contagem, entre outras. Já visão computacional normalmente é usada para referir a processos mais complexos tal como o sistema artificial de visão de um robô ou de um veículo autônomo. Contudo, muitos autores não faz a distinção entre processamento de imagens, análise de imagens e visão computacional. Neste trabalho como usaremos técnicas clássicas de realce e de análise de imagens, também não focaremos nesta distinção, usando assim termo processamento digital de imagens.

\sergio{

O processamento digital de imagens pode ser aplicado em áreas como medicina para auxílio ao diagnóstico médico através de imagens, na arqueologia para fazer restauração de imagens, na biologia para  a análise imagens microscópicas, na área de agricultura para detecção de doenças em plantas, fazer estimativas de produtividade agrícola e análise de vigor vegetativo, na segurança pública para controle de acesso, entre várias outras outras áreas.

Este trabalho foca na contagem de plantas, o que tem se tornado uma possibilidade real de automação computacional devido ao surgimento de satélites de alta resolução, de veículos aéreos de baixo custo usados para o propósito de monitoramento, de computadores de alto poder de processamento capaz de analisar imagens de satélites de alta resolução em segundos, e de variados técnicas computacionais que podem ser empregadas para a contagem de plantas. Na seção a seguir apresentamos a motivação para esta pesquisa tanto do ponto de vista dos benefícios da aplicação, quanto do ponto de vista científico.}
 

\section{Motivação}

\sergio{Contagem de plantas é importante para gerenciamento agrícola e de florestas, provendo informações precisas para planejamento de irrigação, aplicação de fertilizantes, estimativa de produtividade, e inventários de quantidade de biomassa e de estoque de carbono. Um acompanhamento temporal de contagem de plantas também pode prover informação para fiscalização e para verificação de mortalidade de plantas, que pode ser um índice importante quando associado à doenças, condições meteorológicas e fatores do ecossistema~\cite{Daliakopoulos2009}.

O trabalho de contagem de árvores é tradicionalmente feito por pesquisas de campo, que são demoradas, tem alto custo e são altamente suscetível a erros. Contudo os avanços dos últimos anos na qualidade de imagens de satélite e de veículos aéreos não-tripulados (VANTs) têm aberto caminho para várias pesquisas de sensoriamento remoto com variados propósitos. Satélites comerciais como Quickbird, Orbview e Ikonos produzem imagens de alta resolução permitindo a detecção, identificação e contagem de objetos na superfície do solo com alta precisão~\cite{Vibha2009, Srestasathiern2014, Li2017, recio2013, franco2013, Gonzalez2007}. VANTs também têm sido amplamente usados e suas principais vantagens em relação a satélites são a resolução de imagens e possibilidade de obter imagens a qualquer dia e horário~\cite{Disperati2007,Kestur2018}. Satélites são críticos em determinadas épocas do ano devido a presença de nuvens. 


Dada a disponibilidade de imagens de sensoriamento remoto, pesquisadores têm concentrado do desenvolvimento de técnicas que possam vir a substituir a análise humana das imagens captadas. Existe uma variedade de abordagens computacionais para a detecção de árvores, tais como casamento de \textit{template}, segmentação por crescimento de regiões, detecção de picos e métodos aprendizado profundo. Métodos de casamento de \textit{template} constroem uma série de modelos para caracterizar os aspectos de árvores, levando em consideração a geometria da copa e propriedades radiométricas. Uma vez construídos os \textit{templates} um procedimento de casamento por janela deslizante é implementado para buscar os melhores casamentos, isto é, os locais de maiores probabilidades de existência de árvores. Métodos de segmentação por crescimento de regiões partem de vários pixels semente e vai agregando novos pixels na vizinhança conforme algum critério de similaridade. Métodos de filtragem máxima local são usados para detectar picos locais, dados por uma alta saturação de verde normalmente encontrada no interior das copas das árvores. Tal procedimento é aplicado por mecanismo de janela deslizante de um tamanho específico. }


\section{Formulação do problema de pesquisa}

\sergio{Dado uma imagem de satélite colorida (RGB) de alta resolução (de ao menos 2m/pixel) de uma determinada área agrícola delimitada, nosso problema consiste em detectar e contar a quantidade de uma determinada planta utilizando técnicas automáticas de visão computacional. Outras possíveis informação que podem ser usadas para a resolução do problema inclui o espaçamento entre plantas e estimativa da área ocupada por cada planta. O resultado gerado pela resolução do problema deve incluir a localização de cada planta, a delimitação da área ocupada por cada planta, e a contagem das plantas. Os resultados gerados serão avaliados por métricas tradicionais de detecção de objetos que são apresentadas no Capítulo~\ref{cap:metodologia} em comparação com imagens de \textit{ground-truth}.} 


\section{Objetivo}

Nesta pesquisa, técnicas de processamento digital de imagens serão aplicadas na área de agricultura, visando compor uma abordagem efetiva para detecção e contagem de plantas com base em imagens de satélite. Para se chegar a abordagem pretende-se comparar técnicas alternativas para as as subtarefas do processo, avaliando estas através de métricas para mensurar a qualidade dos resultados obtidos.


\section{Metodologia}

\sergio{Este trabalho fará uma investigação experimental de técnicas de processamento de imagens com o objetivo de desenvolver e comparar metodologias para a detecção e contagem de plantas. Assim a metodologia de condução da pesquisa envolve a pesquisa, estudo e experimentação de variadas técnicas candidatas para compor metodologias para a resolução do problema. Devido a existirem várias formas distintas de se tratar o problema, conforme foi mencionado no Capítulo~\ref{}, de trabalhos correlatos, este trabalho foca na experimentação de comparação de abordagens baseadas em segmentação, tal como por crescimento de regiões, e baseados na detecção de picos. Assim, métodos baseados em \textit{template} e em aprendizado profundo não serão investigados neste estudo. 

Dentro do escopo de investigação delimitado as sub-tarefas do problema envolvem:
\begin{enumerate}
    \item Detecção de vegetação: basicamente consiste da distinção entre os pixels de vegetação e de não-vegetação (solo, edificações, leitos d'água, estradas, entre outros). Para isto é tradicional na literatura o emprego de índices de vegetação. Basicamente um índice de vegetação~\cite{} é um cálculo sobre os valores espectrais de pixels, com o propósito de diferenciar pixels de vegetação de pixels de não-vegetação. Um índice de vegetação toma como entrada uma imagens colorida ou multiespectral e retorna uma imagem em níveis de cinza onde pixels que contém uma maior evidência de que seja vegetação, terá valores mais altos. Como o interesse final desse etapa é fazer um distinção entre pixels de vegetação e não vegetação a imagem do índice de vegetação é binarizada. Existem vários métodos de binarização, sendo o mais tradicionais o método de Otsu~\cite{} e métodos de binarização local, tal como Sauvola~\cite{}. 
    
    \item Processamento morfológico: após a binarização é comum a existência regiões desconexas de uma mesmo objeto, objetos distintos conectados, e de pontos e regiões que não correspondem os objetos de interesse, tal como manchas de ervas daninhas. Para amenizar este problemas é comum o emprego de técnicas que fazem uma análise e correção de forma, denominadas de técnicas de morfologia matemática. Estas servem, entre outras coisas, para conexão de objetos, separação de objetos conectados, e eliminação de pequenas regiões.   
    
\end{enumerate}


As metodologias desenvolvidas serão avaliadas através de métricas tradicionais de detecção de objetos através de medidas que exploram os conceitos de falsos positivos, falsos negativos e verdadeiros positivos com base em imagens de ground-truth.}



\section{Principais resultados e contribuições iniciais}

Nesta pesquisa desenvolvemos duas metodologias para a contagens de árvores a partir de imagens de satélite de alta resolução. As metodologias consiste de um processamento comum à ambas: conversão das imagens para níveis de cinza, binarização, tratamento morfológico e aplicação da transformada da distância. Numa segunda etapa do processo comparamos segmentação por inundação (\textit{watershed}) com a abordagem de detecção de picos através do método de filtragem máxima local. Na terceira fase é feita a rotulação dos objetos identificados, onde o número de rótulos corresponderá aos número de árvores estimado. 


\section{Obstáculos Esperados e Desafios Iniciais}

%Conforme destacado na literatura, a contagem de plantas com base em imagens aéreas, coletadas por satélites ou veículos aéreos não-tripulados (VANTs), é importante sob vários aspectos. Por exemplo, \citeonline{Daliakopoulos2009} e \citeonline{Dorj2017} utilizaram a contagem de plantas com base em imagens para estimar o rendimento de culturas.
 
%\citeonline{Daliakopoulos2009} desenvolveu uma metodologia baseada em visão computacional, para monitorar o número de árvores com o passar do tempo. Os autores argumentam que essas contagens ao longo do tempo, pode resultar em índices importantes quando associados a doenças de plantas, condições climáticas e comportamento do ecossistema.

%Segundo \citeonline{Reis2007}, foi resolvido usar abordagens para contagem de plantas para resolver o problema de inventários florestais. Segundo os autores, os dados de inventários florestais têm sido coletados principalmente por pesquisas de campo, que são dispendiosas e demoradas, e muita das vezes com erros devido a falta de treinamento da equipe de campo.

Conforme reportado na literatura, apesar das variados benefícios de análises, existe vários desafios para o desenvolvimento de métodos altamente precisos de contagem de plantas. Entre estes desafios estão a presença de nuvens que pode resultar em oclusões nas imagens capturadas, problemas entre a resolução de imagem e a dimensão da copa das árvores \cite{Srestasathiern2014}, além de problemas de sobreposição da copa das árvores \cite{Disperati2007}.

Segundo \citeonline{Daliman2016}, a análise de árvores individuais com base em imagens de sensoriamento remoto é um problema complexo, pois há variações de tamanho, forma e resposta espectral da copa das árvores. Com isso, o que é detectado como um único objeto pode de fato, em alguns casos corresponder a um grupo de filiais da mesma árvore ou um grupo de árvores. 

Outros desafios para o desenvolvimento de métodos de contagem de plantas, de acordo com os estudos de \citeonline{Daliman2016}, é que poderá haver erros de detecções de árvores, e isso ocorre por causa da proximidade de árvores vizinhas, árvores que estão localizadas sob outras árvores, árvores à sombra ou árvores que tem baixo contraste espectral com o fundo. Apesar desses problemas, é de extrema importância fazer a detecção devida as análises agrícolas e aplicações subsequentes.

\sergio{Apontar as dificuldades e limitações dos resutados iniciais} 

\section{Organização da monografia}

O restante desse trabalho está dividido da seguinte forma: no Capítulo 2 mostra os trabalhos correlacionados;
Capítulo 3...
\colorbox{yellow}
{
   Terminar de arrumar essa seção
}