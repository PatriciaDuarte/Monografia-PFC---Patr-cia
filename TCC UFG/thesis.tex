% ------------------------------------------------------------------------
% ------------------------------------------------------------------------
% UFGRC: Modelo de Trabalho Acadêmico em conformidade com 
% ABNT NBR 14724:2011: Informação e documentação - Trabalhos acadêmicos -
% Apresentação
% ------------------------------------------------------------------------
% ------------------------------------------------------------------------

% Opções: 
%   Tipo do trabalho     = tcc1/tcc2
%   Situação do trabalho = pre-defesa/pos-defesa
% -- opções do pacote babel --
% Idioma padrão = brazil
	%english,			% idioma adicional para hifenização
	%brazil		% o último idioma é o PRINCIPAL do documento
\documentclass[tcc2, pos-defesa, english, brazil]{packages/ufgrc}
% ---------------------------------------------------------------------------
% Pacotes Opcionais
% ---------------------------------------------------------------------------
\usepackage{rotating}   % Usado para rotacionar o texto
\usepackage[all,knot,arc,import,poly]{xy}   % Pacote para desenhos gráficos
\usepackage{color}
% Este pacote pode conflitar com outros pacotes gráficos como o ``pictex''
% Então é necessário usar apenas um dos pacotes conflitantes
\newcommand{\VerbL}{0.52\textwidth}
\newcommand{\LatL}{0.42\textwidth}

\newcommand\sergio[1]{{\color{red}#1}}
\newcommand\patricia[1]{{\color{green}#1}}
% ---------------------------------------------------------------------------


% ---
% Informações de dados para CAPA e FOLHA DE ROSTO
% ---
% Tanto na capa quanto nas folhas de rosto apenas a primeira letra da primeira palavra (ou nomes próprios) devem estar em letra maiúscula, todas as demais devem ser em letra minúscula.
\titulo{Estudo de técnicas de análise de imagens para contagem de plantas}
\autor[Silva, P.D.d.]{Patrícia Duarte da Silva}
\genero{F} % Gênero do autor (M = Masculino / F = Feminino)
\orientador[Orientador]{Prof. Dr.}{Sérgio Francisco da Silva}
%\coorientador{Prof. Dr.}{Fulano de Tal}
\data{03}{6}{2019} % Data da defesa
% ---

% Membros da banca examinadora
% - O primeiro membro será automaticamente o orientador
% - Caso haja coorientador, este será o segundo membro
% Nome dos demais membros e suas instituições
\membrobanca{Fulano de Tal}{Instituição do Fulano de Tal}
\membrobanca{Ciclano de Tal}{Instituição do Ciclano de Tal}

% ---
% RESUMOS
% ---

% Resumo em PORTUGUÊS
% conter no máximo 500 palavras
% conter no mínimo 1 e no máximo 5 palavras-chave (obrigatoriamente separadas por vírgula)
\textoresumo[brazil]{
   A automação da tarefa de contagem de plantas a partir de imagens de sensoriamento remoto tem se tornado importante para variados diagnósticos agrícolas tal como previsão de produtividade, mapeamento de uso da terra, geração de índices pelo acompanhamento temporal de mortalidade de plantas que podem ser associados a doenças e condições climáticas, entre vários outros. Pesquisas recentes têm desenvolvido métodos de análise de imagens e aplicado estes para a contagem de algumas espécies de plantas tal como palmeiras, oliveiras, citros e eucaliptus. Nesta pesquisa pretendemos investigar e desenvolver uma abordagem computacional para a contagem de jabuticabeiras a partir de imagens de satélite. Pretende-se desenvolver uma abordagem que confie em métodos clássicos de processamento de imagens, como índices de vegetação, limiarização, morfologia matemática, caracterização de objetos e classificação. Espera-se que os resultados obtidos tragam contribuições tanto para área de processamento de imagens quanto para a área de análise agrícola. 
    }{Contagem de plantas, processamento de imagens, imagens de satélite.}


% resumo em INGLÊS
% conter no máximo 500 palavras
% conter no mínimo 1 e no máximo 5 palavras-chave (obrigatoriamente separadas por vírgula)
\textoresumo[english]{
The automation of the plant counting task from remote sensing images has become important for a variety of agricultural diagnoses such as productivity prediction, land use mapping, generation of indices by temporal monitoring of plant mortality that can be associated with diseases and climatic conditions, among several others. Recent research has developed methods of image analysis and applied these for counting some species of plants such as palm trees, olive trees, citrus and eucalyptus. In this research we intend to investigate and develop a computational approach to jabuticaba counting from satellite images. It is intended to develop an approach that relies on classical methods of image processing, such as vegetation indexes, thresholding, mathematical morphology, object characterization and classification. It is hoped that the results obtained will bring contributions to both the image processing area and the agricultural analysis area.
    }{Contagem de plantas, processamento de imagens, imagens de satélite.}
% ---
% Configurações de aparência do PDF final
% ---
\hypersetup{
	colorlinks=true     % false: boxed links; true: colored links
}
% --- 

% ----------------------------------------------------------
% ELEMENTOS PRÉ-TEXTUAIS
% ----------------------------------------------------------

% Inserir a ficha catalográfica
%\incluifichacatalografica*{tex/pre-textual/fichaCatalografica.pdf}
\incluifichacatalografica

% DEDICATÓRIA / AGRADECIMENTO / EPÍGRAFE
\textodedicatoria*{tex/pre-textual/dedicatoria}
\textoagradecimentos*{tex/pre-textual/agradecimentos}
\textoepigrafe*{tex/pre-textual/epigrafe}

% Inclui a lista de figuras
\incluilistadefiguras

% Inclui a lista de tabelas
\incluilistadetabelas

% Inclui a lista de quadros
\incluilistadequadros

% Inclui a lista de algoritmos
\incluilistadealgoritmos

% Inclui a lista de códigos
\incluilistadecodigos

% Inclui a lista de siglas e abreviaturas
\incluilistadesiglas

% Inclui a lista de símbolos
\incluilistadesimbolos

% ----
% Início do documento
% ----
\begin{document}

% ----------------------------------------------------------
% ELEMENTOS TEXTUAIS
% ----------------------------------------------------------
\textual

\chapter{Introdução}
\label{chapter:introducao}
% Comando simples para exibir comandos Latex no texto
%\newcommand{\comando}[1]{\textbf{$\backslash$#1}}

\section{Considerações iniciais}

Segundo \cite{Gonzales2010}, a área de processamento digital de imagens, que se refere ao processamento de imagens por um computador digital, está em expansão e possui aplicações em duas amplas categorias: (1) o aprimoramento de informações pictóricas para interpretação humana; e (2) a análise automática por computador para extrair informações de uma cena. 

Historicamente, a área de processamento digital de imagens começou a ser usada na década de 20 no século passado. Inicialmente ela era usada para melhorar a qualidade de transmissões de televisão, e com o passar do tempo, as técnicas foram aprimoradas para essa aplicação. Três décadas mais tarde, a área voltou a se expandir rapidamente, por decorrência do lançamento de computadores digitais de grande porte para uso em programas espaciais. Nesse período, usavam-se diversas técnicas para o realce e a restauração de imagens de programas espaciais, como nas expedições tripuladas da série Apollo, por exemplo. Essas duas tecnologias, realce/restauração de imagens e computadores digitais, andam juntas atualmente, e formam a base para a área de processamento digital de imagens.


Conforme \cite{Gonzales2010}, não há um consenso entre autores sobre onde termina o processamento de imagens e onde começam áreas relacionadas, tais como análise de imagens e visão computacional. Algumas vezes, a distinção é feita por definir processamento de imagens como a área na qual, ambas a entrada e a saída do processo são imagens. A compreensão do conteúdo de imagens, é denominada de análise de imagens ou de visão computacional, dependendo do nível de complexidade contido. Análise de imagens, normalmente é usada para se referir a extração de informações predefinidas a partir de uma imagem, tal como detecção de um objeto, classificação, contagem, entre outras. A visão computacional, é usada para referir a processos mais complexos, tal como o sistema artificial de visão de um robô ou de um veículo autônomo. Contudo, muitos autores não faz a distinção entre processamento de imagens, análise de imagens e visão computacional. Neste trabalho, como usaremos técnicas clássicas de realce e de análise de imagens, também não focaremos nesta distinção, usando assim o termo processamento digital de imagens ou simplesmente processamento de imagens.

O processamento digital de imagens pode ser aplicado em áreas como medicina, para auxílio ao diagnóstico médico através de imagens, na arqueologia para fazer restauração de imagens, na biologia para a análise de imagens microscópicas, na área de agricultura para detecção de doenças em plantações, para fazer estimativas de produtividade agrícola e análise de vigor vegetativo, na segurança pública para controle de acesso, entre várias outras áreas. Este trabalho, foca na detecção e contagem de árvores, o que tem se tornado uma possibilidade real de automação computacional, devido ao surgimento de satélites de alta resolução, de veículos aéreos de custo acessível usados para o propósito de monitoramento, de computadores de alto poder de processamento capaz de analisar imagens de satélites de alta resolução em segundos, e de variadas técnicas computacionais que podem ser empregadas para a contagem de árvores. Na seção a seguir, apresentamos a motivação para esta pesquisa, tanto do ponto de vista dos benefícios da aplicação, quanto do ponto de vista científico.
 

\section{Motivação}

Detecção e contagem de árvores é importante para gerenciamento agrícola e de florestas, provendo informações precisas para planejamento de irrigação, aplicação de fertilizantes, estimativa de produtividade, inventários de quantidade de biomassa e de estoque de carbono. Um acompanhamento temporal de contagem de árvores também pode prover informação para fiscalização e para verificação de mortalidade de árvores, que pode ser um índice importante quando associado à doenças, pragas, condições meteorológicas e fatores do ecossistema~\cite{Daliakopoulos2009}.

O trabalho de contagem de árvores é tradicionalmente feito por pesquisas de campo que são demoradas, têm alto custo e são altamente suscetíveis a erros. Contudo os avanços dos últimos anos na qualidade de imagens de satélite e o surgimento de veículos aéreos não-tripulados (VANTs) de custo acessível têm aberto caminho para várias pesquisas de sensoriamento remoto com variados propósitos. Satélites comerciais como Quickbird, Orbview e Ikonos produzem imagens de alta resolução permitindo a detecção, identificação e contagem de objetos na superfície do solo com alta precisão~\cite{Vibha2009, Srestasathiern2014, Li2017, recio2013, franco2013, Gonzalez2007}. VANTs também têm sido amplamente usados, e suas principais vantagens em relação a satélites são a resolução de imagens e possibilidade de obter imagens a qualquer dia e horário~\cite{Disperati2007,Kestur2018}. Satélites são críticos em determinadas épocas do ano devido a presença de nuvens. 


Dada a disponibilidade de imagens de sensoriamento remoto, pesquisadores têm concentrado no desenvolvimento de técnicas que possam vir a substituir a análise humana das imagens captadas. Existe uma variedade de abordagens computacionais para a detecção de árvores, tais como casamento de \textit{template}, segmentação por crescimento de regiões, detecção de picos e métodos de aprendizado profundo. Métodos de casamento de \textit{template} constroem uma série de modelos para caracterizar os aspectos de árvores, levando em consideração a geometria da copa e propriedades radiométricas. Uma vez construídos os \textit{templates} um procedimento de janela deslizante é implementado para buscar os melhores casamentos, isto é, os locais de maiores probabilidades de existência de árvores. Métodos de segmentação por crescimento de regiões partem de vários pixels semente e vai agregando novos pixels na vizinhança conforme algum critério de similaridade. Métodos de filtragem máxima local são usados para detectar picos locais, dados por uma alta saturação de verde normalmente encontrada no interior das copas das árvores. Tal procedimento é aplicado por mecanismo de convolução usando uma janela deslizante de um tamanho específico dada pela discretização de uma função gaussiana. Métodos de aprendizagem profunda usando a abordagem convolucional têm sido empregados com sucesso em várias tarefas de visão computacional. Estes métodos aprendem uma extração de características através de uma representação interna de pesos da rede neural, que melhor classificam os dados. No Capítulo~\ref{chapter:correlatos} serão apresentadas, as principais pesquisas de cada uma das categoria citadas, para a detecção e contagem de árvores.


\section{Formulação do problema de pesquisa}

Dado uma imagem de satélite colorida (RGB) de alta resolução (de ao menos 2m/pixel) de uma área agrícola delimitada, nosso problema consiste em detectar e contar a quantidade de árvores utilizando técnicas automáticas de processamento de imagens. Outras possíveis informações que podem ser usadas para a resolução do problema, inclui o espaçamento entre árvores e estimativa da área ocupada pela copa de cada árvore. O resultado gerado pela resolução do problema incluirá a localização de cada árvore e/ou a delimitação da área ocupada por sua copa e a contagem das árvores. Pretende-se avaliar os resultados gerados gerados ao final deste trabalho com base em métricas tradicionais de detecção de objetos, tal como precisão, sensitividade, especificidade, entre outras  que são apresentadas no Capítulo~\ref{chapter:desenvolvimento}, em comparação com imagens de \textit{ground-truth} geradas por anotação manual.


\section{Objetivo}

Nesta pesquisa, técnicas de processamento digital de imagens serão aplicadas na área de agricultura, visando compor uma abordagem efetiva para detecção e contagem de árvores com base em imagens de satélite. Para se chegar a abordagem, pretende-se comparar técnicas alternativas para as subtarefas do processo, avaliando estas, através de métricas para mensurar a qualidade dos resultados obtidos.


\section{Metodologia}
As imagens serão coletadas pela API Python do Google Maps. Será aplicado o processamento que consiste de índices de vegetação, binarização, tratamento morfológico e aplicação da transformada da distância. Posteriormente, serão comparados  métodos clássicos de detecção de objetos. Até o momento experimentamos \textit{Watershed} e da filtragem máxima local. Por fim, será feita a rotulação dos objetos identificados, onde o número de rótulos serão o número de árvores estimadas.


\section{Principais resultados e contribuições iniciais}

Até o momento foram desenvolvidas duas metodologias para a contagens de árvores a partir de imagens de satélite de alta resolução. As metodologias consistem de um processamento comum à ambas: conversão das imagens para níveis de cinza, binarização, tratamento morfológico e aplicação da transformada da distância. Na segunda etapa do processo, comparamos segmentação por inundação (\textit{watershed}) com a abordagem de detecção de picos através do método de filtragem máxima local. Na terceira fase, é feita a rotulação dos objetos identificados, onde o número de rótulos corresponderá aos número de árvores estimado. Para PFC2 pretende-se comparar o desempenho de índices de vegetação, ajustar adequadamente os parâmetros envolvidos em cada metodologia, desenvolver uma base de \textit{ground-truth} e validar as metodologias com métricas objetivas de detecção de objetos, tais como sensitividade, especificidade, entre outras. 


\section{Obstáculos esperados e desafios iniciais}

%Conforme destacado na literatura, a contagem de árvores com base em imagens aéreas, coletadas por satélites ou veículos aéreos não-tripulados (VANTs), é importante sob vários aspectos. Por exemplo, \citeonline{Daliakopoulos2009} e \citeonline{Dorj2017} utilizaram a contagem de árvores com base em imagens para estimar o rendimento de culturas.
 
%\citeonline{Daliakopoulos2009} desenvolveu uma metodologia baseada em visão computacional, para monitorar o número de árvores com o passar do tempo. Os autores argumentam que essas contagens ao longo do tempo, pode resultar em índices importantes quando associados a doenças de árvores, condições climáticas e comportamento do ecossistema.

%Segundo \citeonline{Reis2007}, foi resolvido usar abordagens para contagem de árvores para resolver o problema de inventários florestais. Segundo os autores, os dados de inventários florestais têm sido coletados principalmente por pesquisas de campo, que são dispendiosas e demoradas, e muita das vezes com erros devido a falta de treinamento da equipe de campo.

Conforme reportado na literatura, apesar dos variados benefícios de análises, existem vários desafios para o desenvolvimento de métodos altamente precisos para a detecção e contagem de árvores. Entre estes desafios, estão a presença de nuvens que pode resultar em oclusões nas imagens capturadas, problemas envolvendo resolução de imagem e a dimensão da copa das árvores \cite{Srestasathiern2014}, além de problemas de sobreposição da copa das árvores \cite{Disperati2007}.

Segundo \cite{Daliman2016}, a análise de árvores individuais com base em imagens de sensoriamento remoto é um problema complexo, pois há variações do tamanho, da forma e da resposta espectral da copa das árvores. Com isso, o que é detectado como objetos únicos podem de fato corresponder à galhos da uma mesma árvore, ou a um grupo de árvores.

Outros desafios para o desenvolvimento de métodos de contagem de árvores, de acordo com o estudo de \cite{Daliman2016}, é que poderá haver erros de detecções de árvores, e isso ocorre por causa da proximidade de árvores vizinhas, árvores que encobrem outras árvores, árvores à sombra ou árvores que tem baixo contraste espectral com o fundo. Apesar desses problemas, é de extrema importância fazer a detecção devido as análises agrícolas e aplicações subsequentes.

No resultados iniciais pode-se perceber os problemas citados por \cite{Daliman2016}, principalmente devido ao fato de sobreposição da copa das árvores. Pretendemos mensurar quantitativamente e propor soluções para amenizar este problema na disciplina de PFC2 este será desenvolvidas imagens que \textit{ground-truth} que permitirá mensurar adequadamente a qualidade dos resultados obtidos.

\section{Organização da monografia}

O restante desse trabalho está dividido da seguinte forma. No Capítulo~\ref{chapter:conceitosPDI} são apresentados os principais conceitos básicos de processamento de imagens necessários para a compreensão das metodologias em desenvolvimento para a contagem de árvores. No Capítulo~\ref{chapter:correlatos} as pesquisas correlatas de contagens de plantas são categorizadas de acordo com as técnicas aplicadas e discutidas. No Capítulo~\ref{chapter:desenvolvimento} são apresentados o desenvolvimento e os resultados iniciais. Por fim, o Capítulo~\ref{chapter:conclusao} apresenta conclusões sobre o desenvolvimento feito na disciplina de PFC1 e o cronograma. 


\chapter{Conceitos de processamento de imagens para detecção e contagem de plantas}
\label{chapter:orientacoes-gerais}
\section{Considerações iniciais}
Neste capítulo serão descritos conceitos básicos de processamento de imagens usados na literatura, para a detecção e contagem de plantas.

\subsection{Índices de vegetação}

Basicamente um índice de vegetação~\cite{Torres2014}, é um cálculo sobre os valores espectrais de pixels, com o propósito de diferenciar os pixels de vegetação por pixels de não-vegetação, que é aplicado no solo, leitos de água, estradas, construções, entre outros. Um índice de vegetação toma como entrada uma imagem colorida ou multiespectral e retorna uma imagem em níveis de cinza, onde os pixels que contém uma maior evidência de que seja vegetação, terá valores mais altos. Ainda, foi revisado vários experimentos de índices de vegetação, com o intuito de evidênciar os pixels que pertencem as copas das árvores. Na equação(~\ref{eq:normRGB}) é mostrada a entrada para os cálculos desses índices, que são os valores dos canais R, G e B do sistema de cor RGB e as combinações dos índices.

\begin{equation}
    r = \frac{R}{R+G+B}; g = \frac{G}{R+G+B}; b = \frac{B}{R+G+B}
    \label{eq:normRGB}
\end{equation}

A seguir são listados os índices de vegetação apresentados em \cite{Torres2014}.

$\textbf{Índice de diferença verde-vermelho normalizado:}$
\begin{equation}
    NGRDI = \frac{G - R}{G + R}
\end{equation}

$\textbf{Excesso de verde:}$
\begin{equation}
    ExG(2) = 2g - r - b
\end{equation}

$\textbf{Índice de cores da vegetação:}$
\begin{equation}
    CIVE = 0.441r - 0.881g + 0.385b + 18.78745
\end{equation}

$\textbf{Vegetativen:}$
\begin{equation} 
    VEG = \frac{g}{r^{a} b^{(1-a)}}  \ com \ a = 0.667 \ em \ sua \ referência %Não sei o porque deste errinho
\end{equation}

$\textbf{Excesso de verde menos excesso de vermelho:}$
\begin{equation}
    ExGR = ExG - ExR = ExG - 1.4r-g
\end{equation}

$\textbf{Índice Woebbecke:}$
\begin{equation}
    WI = \frac{g-b}{r-g}
\end{equation}

$\textbf{Combinação 1:}$
\begin{equation}
    COM(1) = 0.25ExG + 0.3ExGR + 0.33CIVE + 0.12VEG
\end{equation}

$\textbf{Combinação 2:}$
\begin{equation}
    COM(2) = 0.36ExG + 0.47CIVE + 0.17VEG
\end{equation}


\subsection{Morfologia Matemática para imagens binárias}

Na área de Biologia a palavra "morfologia" geralmente lida com a forma e a estrutura de animais e plantas. Usamos esta palavra com o mesmo contexto na morfologia matemática, como uma ferramenta para extrair componentes de imagem que são úteis na representação e descrição da forma da região. A teoria dos conjuntos é a linguagem base da morfologia matemática. 
Em imagens binárias especialmente, seus conjuntos contém elementos do espaço inteiro $Z^{2}$, em que cada elemento de um conjunto é uma tupla (vetor 2-D), cujas coordenadas são de um pixel de objeto (normalmente em primeiro plano) na imagem. \cite{Gonzales2010}

Na área de processamento digital de imagens, a morfologia emprega dois tipos de conjuntos de pixels: objetos ($A$) e elementos estruturantes ($B$). Os objetos são conjuntos de pixels dos objetos de interesse, já os elementos estruturantes são especificados para fazer análise e processamentos dos objetos de interesse, e normalmente levam em conta as características geométricas destes. 

\subsubsection{Erosão}
Sendo $A$ e $B$ conjuntos em $Z^{2}$, a erosão de $A$ por $B$, denota $A \ominus B$, é definida como:
\begin{equation}
     A\ominus B = \left\{ z|(B)_{z} \subseteq A\right\}
\end{equation}

onde $A$ é um conjunto de pixels de objetos, $B$ é um elemento estruturante e $z$ são translações (deslocamentos) de $B$ ao longo do plano da imagem. Em palavras, esta equação indica que a erosão de $A$ por $B$ é o conjunto de todos os pontos $z$, de modo que $B$, transladado por $z$, está contido em $A$. \sergio{FAZER EXEMPLO DE EROSÃO E DIZER O QUE ELA FAZ NA PRÁTICA}


\subsubsection{Dilatação}
Novamente sendo $A$ e $B$ conjuntos em $Z^{2}$, a dilatação de $A$ por $B$, denotada como $A \oplus B$, é definida como: 
\begin{equation}
    A \oplus B = \left \{ Z | (\hat{B})_{z}\cap A \neq \varnothing \right \}
\end{equation}
sendo $\hat B $ a reflexão de $B$. Em resumo, a dilatação de $A$ por $B$ é o conjunto de todos os deslocamentos z, de modo que os elementos de $\hat B$ se sobrepõem a pelo menos um elemento de $A$.
\sergio{FAZER EXEMPLO DE EROSÃO E DIZER O QUE ELA FAZ NA PRÁTICA}

\subsubsection{Abertura}

A abertura do conjunto $A$ pelo elemento estruturante $B$, indicado por $A \circ  B$, é definida como
\begin{equation}
    A \circ B = \left ( A \ominus  B \right )\oplus  B
\end{equation}

Assim, a abertura $A$ por $B$ é a erosão de $A$ por $B$, seguida de uma dilatação do resultado por $B$. 
A abertura geralmente suaviza o contorno de um objeto, quebra os istmos estreitos e elimina saliências finas.

\subsubsection{Fechamento}
O fechamento do conjunto $A$ pelo elemento estruturante $B$, denominado $A \bullet B$, é definido como
\begin{equation}
    A \bullet  B = \left ( A \oplus  B \right )\ominus  B
\end{equation}
ou seja, o fechamento de $A$ por $B$ é simplesmente a dilatação de $A$ por $B$, seguido pela erosão do resultado por $B$. 
O fechamento tende a suavizar seções de contornos, mas, ao contrário da abertura, geralmente funde quebras estreitas e golfos finos e longos, elimina pequenos orifícios e preenche lacunas no contorno.

\subsection{Rotulação de Componentes conectados}
Segundo \cite{Gonzales2010}, a capacidade de extrair componentes conectados de uma imagem binária é central para muitos aplicações automatizadas de análise de imagem. Seja $A$ um conjunto de pixels de objetos que consiste em um ou mais componentes conectados e forme uma imagem $X_{0}$ (do mesmo tamanho que $I$, a imagem que contém $A$) cujos elementos são 0's (valores do plano de fundo), exceto em cada local conhecido por corresponder a um ponto em cada componente conectado em A, que definimos como 1 (valor de pixel de objeto). O objetivo é começar com $ X_ {0}$ e encontrar todos os componentes conectados em I. Um procedimento iterativo que a seguir realiza isso:
\begin{equation}
    X_{k} = (X_{k-1} \oplus B) \cap I \qquad \qquad k = 1,2,3...
\end{equation}

onde B é o elemento estruturante. O procedimento termina quando $X_{k} = X_{k-1}$, com $X_{k}$ contendo todos os componentes conectados dos pixels em primeiro plano na imagem. 

\subsection{Método de Otsu}

Segundo \cite{Gonzales2010}, o método é ótimo no sentido de maximizar a variação entre classes, uma medida bem conhecida usada na análise estatística discriminante. A idéia básica é que as classes com limiares apropriados devem ser distintas com respeito aos valores de intensidade de seus pixels e, inversamente, que um limite que ofereça a melhor separação entre as classes em termos de seus valores de intensidade seja o melhor (ótimo) limite. Além de sua otimização, o método de Otsu tem a propriedade importante de que ele se baseia inteiramente em cálculos realizados no histograma de uma imagem, uma matriz 1-D facilmente obtida.

Seja {0, 1, 2 ... L-1} denotar o conjunto de L níveis de intensidade inteira distintos em uma imagem digital de tamanho M$\times$N pixels, e seja $n_{i}$ o número de pixels com intensidade i.
O número total, MN, de pixels na imagem é MN = $n_{0} + n_{1} + n_{2} + + n_{L-1}$. O histograma normalizado possui componentes $p_{i} = n_{i}$ / MN, dos quais se segue que
\begin{equation}
    \sum_{i=0}^{L-1} p_{i}=1 \qquad \qquad p_{i}\geq 0
\end{equation}

Agora, suponha que selecionamos um limite T(k) = k, 0 < k < L-1, e use-o para limitar a imagem de entrada em duas classes, $c_{1}$ e $c_{2}$, onde $c_{1}$ consiste em todos os pixels na imagem com valores de intensidade no intervalo [0,k] e $c_{2}$ consiste em pixels com valores no intervalo $[k+1, L-1]$. Usando esse limite, a probabilidade, $P_{1}(k)$, de que um pixel é atribuído à classe $c_{1}$ (ou seja, com limite para) é dada pela soma cumulativa
\begin{equation}
    p_{1}(k) = \sum_{i=0}^{k} p_{i}
\end{equation}

Visto de outra maneira, essa é a probabilidade da classe $c_{1}$ ocorrer. Por exemplo, se definirmos k=0, a probabilidade da classe $c_{1}$ ter pixels atribuídos a ela é zero. Da mesma forma, a probabilidade de ocorrência da classe $c_{2}$ é
\begin{equation}
    p_{2}(k) = \sum_{i=k+1}^{L-1} p_{i} = 1- p_{1}(k)
\end{equation}

O valor médio da intensidade dos pixels em $c_{1}$ é
\begin{equation}
    \begin{split}
    m_{1}(k) = \sum_{i=0}^{k}iP(i/c_{1}) = \sum_{i=0}^{k} iP(c_{1}/i)P(i)/P(c_{1}) \\
    = \frac{1}{P_{1}(k)}\sum_{i=0}^{k} ip_{i}
    \end{split}
\end{equation}

onde $P_{1}$(k) é dado pela Eq.(2.16). O termo P(i/$c_{1}$) na Eq. (2.18) é a probabilidade do valor de intensidade i, dado que i provém da classe $c_{1}$. O termo mais à direita na primeira linha da equação segue da fórmula de Bayes:
\begin{equation}
P(A/B) = P(B/A)P(A)/P(B)    
\end{equation}

A segunda linha decorre do fato de que P($c_{1}$/i), a probabilidade de $c_{1}$ dado i, é 1 porque estamos lidando apenas com valores de i da classe $c_{1}$. Além disso, P(i) é a probabilidade do i-ésimo valor, que é o i-ésimo componente do histograma, $p_{i}$. Finalmente, P($c_{1}$) é a probabilidade da classe $c_{1}$ que, da Eq.(2.16), é igual a $P_{1}$(k).
Da mesma forma, o valor médio da intensidade dos pixels atribuídos à classe $c_{2}$ é
\begin{equation}
    \begin{split}
        m_{2}(k) = \sum_{i=k+1}^{L-1}iP(i/c_{2}) \\
        = \frac{1}{P_{2}(k)} \sum_{i=k+1}^{L-1} i p_{i}
    \end{split}
\end{equation}

A média acumulada(intensidade média) até o nível k é dada por
\begin{equation}
    m(k) = \sum_{i=0}^{k} i p_{i}
\end{equation}

e a intensidade média de toda a imagem (ou seja, a média global) é dada por
\begin{equation}
    m_{G} = \sum_{i=0}^{L-1} ip_{i}
\end{equation}

A validade das duas equações a seguir pode ser verificada por substituição direta dos resultados anteriores:
\begin{equation}
    P_{1}m_{1} + P_{2}m_{2} = m_{G}
\end{equation}

e
\begin{equation}
    P_{1} + P_{2} = 1
\end{equation}

onde omitimos os ks temporariamente em favor da clareza notacional.

Para avaliar a eficácia do limiar no nível k, é usada a medida normalizada, sem dimensão
\begin{equation}
    \eta  = \frac{\sigma _{B}^{2}}{\sigma_{G}^{2}}
\end{equation}

onde $\sigma_{G}^{2}$ é a variação global, ou seja, a variação de intensidade de todos os pixels da imagem,
\begin{equation}
\sigma_{G}^{2} = \sum_{i=0}^{L-1}(i - m_{G})^{2} p_{i}    
\end{equation}

e $\sigma_{B}^{2}$ é a variação entre classes, definida como
\begin{equation}
    \sigma_{B}^{2} = P_{1}(m_{1} - m_{G})^{2} + P_{2}(m_{2} -m_{G})^{2}
\end{equation}

Essa expressão também pode ser escrita como
\begin{equation}
    \begin{split}
             \sigma_{B}^{2} = P_{1}P_{2}(m_{1} - m_{2})^{2} \\
            = \frac{(m_{G}P_{1} - m)^{2}}{P_{1}(1-P_{1})}
    \end{split}
\end{equation}

A primeira linha desta equação segue das Equações(2.23), (2.24) e (2.27). A segunda linha segue das Equações (2.17) a (2.22). Essa forma é um pouco mais eficiente computacionalmente porque a média global, $m_{G}$, é calculada apenas uma vez; portanto, apenas dois parâmetros, $m_{1}$ e $P_{1}$, precisam ser computados para qualquer valor de k.
A primeira linha na Equação(2.28) indica que quanto mais as duas médias $m_{1}$ e $m_{2}$ estiverem uma da outra, maior $\sigma_{B}^{2}$  será, implicando que a variação entre classes é uma medida de separabilidade entre classes. Como $\sigma_{G}^{2}$ é uma constante, segue-se que $\eta$ também é uma medida de separabilidade, e maximizar essa métrica é equivalente a maximizar $\sigma_{B}^{2}$. O objetivo, então, é determinar o valor do limite, k, que maximize a variação entre as classes, conforme declarado anteriormente. Observe que a Equação (2.25) assume implicitamente que $\sigma_{G}^{2}$ > 0. Essa variação pode ser zero somente quando todos os níveis de intensidade da imagem forem
o mesmo, o que implica a existência de apenas uma classe de pixels. Por sua vez, isso significa que $\eta$ = 0 para uma imagem constante porque a separabilidade de uma única classe de si mesma é zero.

Reintroduzindo k, temos os resultados finais:
\begin{equation}
    \eta(k) = \frac{\sigma_{B}^{2}(k)}{\sigma_{G}^{2}}
\end{equation}

e

\begin{equation}
    \sigma_{B}^{2}(k) = \frac{[m_   {G}P_{1}(k) - m(k)]^2}{P_{1}(k)[1-P_{1}(k)]}
\end{equation}

Então, o limite ideal é o valor, k*, que maximiza $\sigma_{B}^{2}$ (k):
\begin{equation}
    \sigma_{B}^{2}(k^{*}) = _{0\leq k \leq L-1}^{max \ \sigma_{B}^{2}(k)}
\end{equation}

Para encontrar $k^{*}$, simplesmente avaliamos essa equação para todos os valores inteiros de k (sujeito à condição 0<$P_{1}$(k)<1) e selecionamos o valor de k que produziu o máximo de $\sigma_{B}^{2}$(k).
Se o máximo existe para mais de um valor de k, é habitual calcular a média dos vários valores de k para os quais $\sigma_{B}^{2}$(k) é máximo.  Avaliando as Equações (2.30) e (2.31) para todos os valores de k é um procedimento computacional relativamente barato, porque o número máximo de valores inteiros que k pode ter é L, que é apenas 256 para imagens de 8 bits.

Uma vez obtido $k^{*}$, a imagem de entrada f(x, y) é segmentada como antes:
\begin{equation}
      g(x,y) = \left \{  \begin{array}{cc}
        1  & if \ f(x,y) > k^{*}\\
        1  & if \ f(x,y) \leq  k^{*} \\
    \end{array} \right \}
\end{equation}

para x = 0,1,2,...,M-1 e y = 0,1,2,...,N-1. Observe que todas as quantidades necessárias para avaliar a Equação(2.30) são obtidos usando apenas o histograma de f(x,y). Além do limite ideal, outras informações sobre a imagem segmentada podem ser extraídas do histograma. Por exemplo, $P_{1}$ ($k^{*}$) e $P_{2}$ ($k^{*}$), as probabilidades da classe avaliadas no limite ideal, indicam as partes das áreas ocupadas pelas classes (grupos de pixels) na imagem em limiar. Da mesma forma, as médias $m_{1} (k^{*})$ e $m_{2} (k^{*})$ são estimativas da intensidade média das classes na imagem original.

Em geral, a medida na Equação(2.29) tem valores na faixa
\begin{equation}
    0 \leq \eta(k) \leq  1
\end{equation}

para valores de k no intervalo [0, L-1]. Quando avaliada no limiar ideal $k^{*}$, essa medida é uma estimativa quantitativa da separabilidade de classes, o que, por sua vez, nos dá uma idéia da precisão de limiar uma determinada imagem com $k^{*}$. O limite inferior na Equação(2.33) é atingível apenas por imagens com um nível de intensidade único e constante. O limite superior é atingível apenas por imagens de dois valores com intensidades
igual a 0 e L-1.


\subsection{Transformada de Distância}
Foi mostrado por \citeonline{Maurer2003}, a métrica de distância $L_{p}$
\begin{equation}
    \Delta (x,y) = \left ( \sum_{i=1}^{k}\mid x_{i} - y_{i} \mid ^p \right )\tfrac{1}{p} \ \ ,
\end{equation}

onde x e y são k-tuplas, $x_{i}$ e $_{i}$  são as i-ésimas coordenadas de x e y, e 1 $\leq$ p $\leq \infty$ . As métricas $L_{1}$, $L_{2}$ e $L_{1}$ são conhecidas como distâncias de Manhattan ou quarteirão, Euclidiana e tabuleiro de xadrez. A seguinte métrica ponderada da distância $L_{p}$ é
\begin{equation}
      \Delta (x,y) = \left ( \sum_{i=1}^{k} \mid w_{i}(x_{i} - y_{i}) \mid ^p \right )\tfrac{1}{p} \ \ ,
\end{equation}

onde $w_{i}$ é o peso das i-ésimas coordenadas de x e y. As métricas de distância: $\Delta$: $R^{k} \times R^{k} \rightarrow R$  satisfazem as seguintes propriedades:

$\textbf{Propriedade 1}$: Definitividade positiva; $\Delta (x,y) = 0 \ \  if \ f \ \ x = y$

$\textbf{Propriedade 2}$: Simetria; $\Delta(x,y) = \Delta(y,x) \ for \ any \ x \ and \ y $

$\textbf{Propriedade 3}$: Desigualdade de triângulo; $\Delta(x,z)\leq  \Delta(x,y) + \Delta(y,z) \ for \ any \ x , \ y, \ and \ z $

$\textbf{Propriedade 4}$: Monotonicidade; Seja x e y duas k-tuplas que diferem apenas nos valores das coordenadas (ou seja, $x_{i}$ = $y_{i}$, i $\neq$ d). Para concretude, assuma que $x_{d}$ < $y_{d}$. Para qualquer $\textbf{u}$ e $\textbf{v}$ de tal forma que 1) $\Delta$(x,$\textbf{u}$) $\leq$ $\Delta$(x,v) e $\Delta$(y,v) < $\Delta$(y,$\textbf{u}$) ou 2) $\Delta$(x,$\textbf{u}$) < $\Delta$(x,v) e $\Delta$(y,v) $\leq$ $\Delta$(y,$\textbf{u}$) detém , $u_{d}$ < $u_{d}$

$\textbf{Propriedade 5}$. Seja x e y duas k-tuplas que diferem apenas nos valores das coordenadas dth (ou seja, $x_{i}$ = $y_{i}$, i $\neq$ d). Seja $\textbf{u}$ e $\textbf{v}$ duas k-tuplas com valores idênticos das coordenadas dth (ou seja, $u_{d}$ = $v_{d}$. Se $\Delta$(x,$\textbf{u}$) $\leq$ $\Delta$(x,v), então $\Delta$(x,$\textbf{u}$) $\leq$ $\Delta$(y,v).


\subsection{Watershed}
Segundo \cite{Gonzales2010}, o conceito de Watershed(bacia hidrográfica) é baseado na visualização de uma imagem em três dimensões, duas coordenadas espaciais versus intensidade. Nessa interpretação “topográfica”, consideramos três tipos de pontos: 1- Pontos pertencentes a um mínimo regional; 2- Pontos nos quais uma gota de água, se colocada no local de qualquer um desses pontos, cairia com certeza em um único mínimo; e 3- Pontos em que a água teria a mesma probabilidade de cair para mais de um mínimo. Para um mínimo regional específico, o conjunto de pontos que satisfazem a condição 2, é chamado de bacia hidrográfica ou bacia hidrográfica desse mínimo. Os pontos que satisfazem a condição 3, formam linhas de crista na superfície topográfica e são chamados de linhas de divisão ou linhas de bacias hidrográficas.

O principal objetivo desse algoritmo, é encontrar as linhas da bacia hidrográfica. 
Uma das principais aplicações, é a extração de objetos quase uniformes (semelhantes a bolhas) do fundo. Regiões caracterizadas por pequenas variações de intensidade têm pequenos valores de gradiente. Assim, na prática, geralmente vemos a segmentação de bacias hidrográficas aplicada ao gradiente de uma imagem, e não a própria imagem. Nesta formulação, os mínimos regionais das bacias hidrográficas se correlacionam muito bem com o pequeno valor do gradiente correspondente aos objetos de interesse.

A construção da barragem do algoritmo de segmentação das bacias hidrográficas, é baseada em imagens binárias, que são membros do espaço 2-D do número inteiro $Z^{2}$. A maneira mais simples de construir barragens que separam conjuntos de pontos binários, é usar a dilatação morfológica.
Primeiramente, para a construção das barragens é aplicado a dilatação. Posteriormente, possui a etapa de inundação n-1, a água derrama de uma bacia para outra, e uma represa deve ser construída para impedir que isso aconteça. $M_{1}$ e $M_{2}$ denotam os conjuntos de coordenadas de pontos em dois mínimos regionais. Em seguida, deixe o conjunto de coordenadas de pontos na bacia hidrográfica associado a esses dois mínimos no estágio n-1 da inundação, sendo denotados por $C_{n-1} (M_{1})$ e $C_{n-1} (M_{2})$.

Dois componentes conectados que se tornaram um único componente, indicam que a água entre as duas bacias hidrográficas, se fundiu na etapa de inundação n. Esse componente  é deixado de ser conectado se for indicado por q. 
Suponha que cada um dos componentes conectados seja dilatado pelo elemento estruturador, sujeito a duas condições: 1- A dilatação deve ser restringida a q (isso significa que o centro do elemento estruturador pode ser localizado apenas nos pontos em q durante a dilatação); e 2- A dilatação não pode ser realizada em pontos que causariam a dilatação dos conjuntos (isto é, se tornariam um único componente conectado). 

É evidente que os únicos pontos em q, que satisfazem as duas condições, descrevem o caminho conectado de um pixel de espessura hachurado. Esse caminho é a barragem de separação desejada no estágio n das inundações. A construção da barragem nesse nível de inundação, é concluída definindo todos os pontos no caminho apenas determinado para um valor maior que o valor máximo de intensidade possível da imagem (por exemplo, maior que 255 para uma imagem de 8 bits). Isso impedirá que a água atravesse a parte da barragem concluída à medida que o nível de inundação aumenta. 

\subsection{Medidas de precisão dos resultados}
Podemos quantificar o desempenho de um algoritmo de correspondência em um determinado limite, contando primeiro o número de correspondências verdadeiras e falsas, e falhas de correspondência, usando as seguintes definições:\cite{Szeliski2010}

\textbf{TP:} verdadeiros positivos, isto é, número de correspondências corretas;

\textbf{FN:} falsos negativos, correspondências que não foram detectadas corretamente;

\textbf{FP:} falsos positivos, correspondências propostas incorretas;

\textbf{TN:} negativos verdadeiros, não correspondências que foram corretamente rejeitadas.


Podemos converter esses números em taxas unitárias, definindo as seguintes quantidades:

\textbf{TPR:} taxa positiva verdadeira,
\begin{equation}
    TPR: \frac{TP}{TP + FN} = \frac{TP}{P};
\end{equation}

\textbf{FPR:} taxa de falso positivos,
\begin{equation}
    FPR: \frac{FP}{FP+TN} = \frac{FP}{N};
\end{equation}

\textbf{PPV:}valor preditivo positivo,
\begin{equation}
    PPV = \frac{TP}{TP+FP} = \frac{TP}{P'};
\end{equation}

\textbf{ACC}: acurácia,
\begin{equation}
    ACC = \frac{TP+TN}{P+N};
\end{equation}

O termo  \textit{precisão} (quantos documentos retornados são relevantes) é usado em vez de PPV e \textit{recall} (qual fração documentos relevantes foi encontrado) é usado no lugar do TPR.
Qualquer estratégia de correspondência específica (em um determinado limite ou configuração de parâmetro) pode ser classificada pelos números TPR e FPR; idealmente, a taxa positiva verdadeira será próxima de 1 e a taxa positiva falsa próxima de 0.

\chapter{Trabalhos correlacionados}
\label{chapter:instalando-abntex}
\input{tex/instalando-abntex}

\chapter{METODOLOGIA}
\label{chapter:config-pre-textual}
\input{tex/config-pre-textual}

\chapter{Conclusão}
\label{chapter:corpos-flutuantes}
\input{tex/corpos-flutuantes}

% ---
% Finaliza a parte no bookmark do PDF, para que se inicie o bookmark na raiz
% ---
\bookmarksetup{startatroot}% 
% ---

% ----------------------------------------------------------
% ELEMENTOS PÓS-TEXTUAIS
% ----------------------------------------------------------
\postextual

% ----------------------------------------------------------
% Referências bibliográficas
% ----------------------------------------------------------
\bibliography{references}

% ---------------------------------------------------------------------
% GLOSSÁRIO
% ---------------------------------------------------------------------

% Arquivo que contém as definições que vão aparecer no glossário
\input{tex/glossario}
% Comando para incluir todas as definições do arquivo glossario.tex
\glsaddall
% Impressão do glossário
\printglossaries

% ----------------------------------------------------------
% Apêndices
% ----------------------------------------------------------

% ---
% Inicia os apêndices
% ---
\begin{apendicesenv}

    \chapter{Códigos}
    \label{chapter:documento-basico}
    \input{tex/appendix/documento-basico}

\end{apendicesenv}
% ---


% ----------------------------------------------------------
% Anexos
% ----------------------------------------------------------

% ---
% Inicia os anexos
% ---

\end{document}