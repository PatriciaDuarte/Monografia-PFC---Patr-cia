% ------------------------------------------------------------------------
% ------------------------------------------------------------------------
% UFGRC: Modelo de Trabalho Acadêmico em conformidade com 
% ABNT NBR 14724:2011: Informação e documentação - Trabalhos acadêmicos -
% Apresentação
% ------------------------------------------------------------------------
% ------------------------------------------------------------------------

% Opções: 
%   Tipo do trabalho     = tcc1/tcc2
%   Situação do trabalho = pre-defesa/pos-defesa
% -- opções do pacote babel --
% Idioma padrão = brazil
	%english,			% idioma adicional para hifenização
	%brazil		% o último idioma é o PRINCIPAL do documento
\documentclass[tcc2, pos-defesa, english, brazil]{packages/ufgrc}
% ---------------------------------------------------------------------------
% Pacotes Opcionais
% ---------------------------------------------------------------------------
% para criar tabela que pode ser quebrado em várias páginas
\usepackage{longtable}
\usepackage{subfigure}
\usepackage{rotating}   % Usado para rotacionar o texto
\usepackage[all,knot,arc,import,poly]{xy}   % Pacote para desenhos gráficos
\usepackage{color}
% Este pacote pode conflitar com outros pacotes gráficos como o ``pictex''
% Então é necessário usar apenas um dos pacotes conflitantes
\newcommand{\VerbL}{0.52\textwidth}
\newcommand{\LatL}{0.42\textwidth}

\newcommand\sergio[1]{{\color{red}#1}}
\newcommand\patricia[1]{{\color{green}#1}}

\usepackage{amsmath}
\DeclareMathOperator*{\argmax}{argmax} % thin space, limits underneath in displays

% ---------------------------------------------------------------------------


% ---
% Informações de dados para CAPA e FOLHA DE ROSTO
% ---
% Tanto na capa quanto nas folhas de rosto apenas a primeira letra da primeira palavra (ou nomes próprios) devem estar em letra maiúscula, todas as demais devem ser em letra minúscula.
\titulo{Estudo de técnicas de processamento de imagens para contagem de árvores em imagens de satélite}
\autor[Silva, P.D.d.]{Patrícia Duarte da Silva}
\genero{F} % Gênero do autor (M = Masculino / F = Feminino)
\orientador[Orientador]{Prof. Dr.}{Sérgio Francisco da Silva}
%\coorientador{Prof. Dr.}{Fulano de Tal}
\data{03}{6}{2019} % Data da defesa
% ---

% Membros da banca examinadora
% - O primeiro membro será automaticamente o orientador
% - Caso haja coorientador, este será o segundo membro
% Nome dos demais membros e suas instituições
\membrobanca{Núbia Rosa da Silva}{Instituto de Biotecnologia}
\membrobanca{Tércio Alberto dos Santos Filho}{Instituto de Biotecnologia}

% ---
% RESUMOS
% ---

% Resumo em PORTUGUÊS
% conter no máximo 500 palavras
% conter no mínimo 1 e no máximo 5 palavras-chave (obrigatoriamente separadas por vírgula)
\textoresumo[brazil]{
    A contagem de árvores é importante para várias aplicações envolvendo manejo, análise, e geração de inventários agrícolas e florestais. Atualmente não há métodos automáticos universais para a contagem de árvores, contudo há tecnologias de imageamento aéreo como satélites de alta resolução de imagem e \textit{drones} de custo acessível, que podem ser usados para capturar imagens. Tem-se também o avanço da computação e o surgimento de variadas técnicas de visão computacional que podem ser aplicadas para a contagem de árvores. Este projeto visa o desenvolvimento e análise de métodos para detecção e contagem de árvores com base em segmentação por inundação e em detecção de picos. Espera-se produzir contribuições científicas e desenvolver métodos de alta precisão para a detecção e contagem de árvores. Nesta pesquisa são propostas duas metodologias de visão computacional para a contagem de árvores a partir de imagens de satélite. As metodologias consistem de um processamento comum à ambas, dado por conversão das imagens para níveis de cinza, binarização, tratamento morfológico e aplicação da transformada da distância. Numa segunda etapa do processo, foi comparado segmentação por inundação (\textit{watershed}) com detecção de picos por filtragem máxima local. Na terceira fase é feita a rotulação dos objetos identificados. A metodologia foi avaliada em imagens de satélite obtidas via API Google Maps de dois tipos de árvores: jabuticabeiras e coqueiros. Os melhores resultados iniciais foram obtidos para filtragem máxima local, com 92.03\% de precisão para a contagem de plantas de jabuticabeiras e 92.88\% para coqueiros.
   
    }{Contagem de árvores, processamento de imagens, imagens de satélite.}


% resumo em INGLÊS
% conter no máximo 500 palavras
% conter no mínimo 1 e no máximo 5 palavras-chave (obrigatoriamente separadas por vírgula)
\textoresumo[english]{
Tree counting is important for many applications involving management, analysis and generation of agricultural and forest inventories. Currently does not exist an automatic universal tree counting methods, but there are aerial imaging technologies such as high resolution satellites and affordable drones that can be used to capture images. Additionally, there is the advancement of computation and the increase of techniques of computer vision that can be applied to tree counting. This project aims to develop and analyze methods for tree detection and counting based on flood segmentation and peak detection. It is expected to produce scientific contributions and to develop high precision methods for tree detection and counting. In this research, two computer vision methodologies for counting trees from satellite images are presented. As the methodologies consist of a common processing, given by the conversion of the images to gray levels, binarization, morphological treatment and application of the distance transformation. In a second phase we compare flooding segmentation (\textit {watershed}) with peak detection by local maximum filtering. In the third phase the detected objects are labeled. The methodology was evaluated in satellite images collected via Google Maps API of two types of trees: jabuticaba and coconut trees. The best results were achieved for maximum local filtration, with 92.03 \% accuracy for jabuticaba tree counts and 92.88 \% for coconut trees counts.
}{Tree counting, image processing, satellite images.}
% ---
% Configurações de aparência do PDF final
% ---
\hypersetup{
	colorlinks=true     % false: boxed links; true: colored links
}
% --- 

% ----------------------------------------------------------
% ELEMENTOS PRÉ-TEXTUAIS
% ----------------------------------------------------------

% Inserir a ficha catalográfica
%\incluifichacatalografica*{tex/pre-textual/fichaCatalografica.pdf}
\incluifichacatalografica

% DEDICATÓRIA / AGRADECIMENTO / EPÍGRAFE
%\textodedicatoria*{tex/pre-textual/dedicatoria}
%\textoagradecimentos*{tex/pre-textual/agradecimentos}
%\textoepigrafe*{tex/pre-textual/epigrafe}

% Inclui a lista de figuras
\incluilistadefiguras

% Inclui a lista de tabelas
\incluilistadetabelas

% Inclui a lista de quadros
%\incluilistadequadros

% Inclui a lista de algoritmos
%\incluilistadealgoritmos

% Inclui a lista de códigos
%\incluilistadecodigos

% Inclui a lista de siglas e abreviaturas
%\incluilistadesiglas

% Inclui a lista de símbolos
%\incluilistadesimbolos

% ----
% Início do documento
% ----
\begin{document}

% ----------------------------------------------------------
% ELEMENTOS TEXTUAIS
% ----------------------------------------------------------
\textual

\chapter{Introdução}
\label{chapter:introducao}
% Comando simples para exibir comandos Latex no texto
%\newcommand{\comando}[1]{\textbf{$\backslash$#1}}

\section{Considerações iniciais}

Segundo \cite{Gonzales2010}, a área de processamento digital de imagens, que se refere ao processamento de imagens por um computador digital, está em expansão e possui aplicações em duas amplas categorias: (1) o aprimoramento de informações pictóricas para interpretação humana; e (2) a análise automática por computador para extrair informações de uma cena. 

Historicamente, a área de processamento digital de imagens começou a ser usada na década de 20 no século passado. Inicialmente ela era usada para melhorar a qualidade de transmissões de televisão, e com o passar do tempo, as técnicas foram aprimoradas para essa aplicação. Três décadas mais tarde, a área voltou a se expandir rapidamente, por decorrência do lançamento de computadores digitais de grande porte para uso em programas espaciais. Nesse período, usavam-se diversas técnicas para o realce e a restauração de imagens de programas espaciais, como nas expedições tripuladas da série Apollo, por exemplo. Essas duas tecnologias, realce/restauração de imagens e computadores digitais, andam juntas atualmente, e formam a base para a área de processamento digital de imagens.


Conforme \cite{Gonzales2010}, não há um consenso entre autores sobre onde termina o processamento de imagens e onde começam áreas relacionadas, tais como análise de imagens e visão computacional. Algumas vezes, a distinção é feita por definir processamento de imagens como a área na qual, ambas a entrada e a saída do processo são imagens. A compreensão do conteúdo de imagens, é denominada de análise de imagens ou de visão computacional, dependendo do nível de complexidade contido. Análise de imagens, normalmente é usada para se referir a extração de informações predefinidas a partir de uma imagem, tal como detecção de um objeto, classificação, contagem, entre outras. A visão computacional, é usada para referir a processos mais complexos, tal como o sistema artificial de visão de um robô ou de um veículo autônomo. Contudo, muitos autores não faz a distinção entre processamento de imagens, análise de imagens e visão computacional. Neste trabalho, como usaremos técnicas clássicas de realce e de análise de imagens, também não focaremos nesta distinção, usando assim o termo processamento digital de imagens ou simplesmente processamento de imagens.

O processamento digital de imagens pode ser aplicado em áreas como medicina, para auxílio ao diagnóstico médico através de imagens, na arqueologia para fazer restauração de imagens, na biologia para a análise de imagens microscópicas, na área de agricultura para detecção de doenças em plantações, para fazer estimativas de produtividade agrícola e análise de vigor vegetativo, na segurança pública para controle de acesso, entre várias outras áreas. Este trabalho, foca na detecção e contagem de árvores, o que tem se tornado uma possibilidade real de automação computacional, devido ao surgimento de satélites de alta resolução, de veículos aéreos de custo acessível usados para o propósito de monitoramento, de computadores de alto poder de processamento capaz de analisar imagens de satélites de alta resolução em segundos, e de variadas técnicas computacionais que podem ser empregadas para a contagem de árvores. Na seção a seguir, apresentamos a motivação para esta pesquisa, tanto do ponto de vista dos benefícios da aplicação, quanto do ponto de vista científico.
 

\section{Motivação}

Detecção e contagem de árvores é importante para gerenciamento agrícola e de florestas, provendo informações precisas para planejamento de irrigação, aplicação de fertilizantes, estimativa de produtividade, inventários de quantidade de biomassa e de estoque de carbono. Um acompanhamento temporal de contagem de árvores também pode prover informação para fiscalização e para verificação de mortalidade de árvores, que pode ser um índice importante quando associado à doenças, pragas, condições meteorológicas e fatores do ecossistema~\cite{Daliakopoulos2009}.

O trabalho de contagem de árvores é tradicionalmente feito por pesquisas de campo que são demoradas, têm alto custo e são altamente suscetíveis a erros. Contudo os avanços dos últimos anos na qualidade de imagens de satélite e o surgimento de veículos aéreos não-tripulados (VANTs) de custo acessível têm aberto caminho para várias pesquisas de sensoriamento remoto com variados propósitos. Satélites comerciais como Quickbird, Orbview e Ikonos produzem imagens de alta resolução permitindo a detecção, identificação e contagem de objetos na superfície do solo com alta precisão~\cite{Vibha2009, Srestasathiern2014, Li2017, recio2013, franco2013, Gonzalez2007}. VANTs também têm sido amplamente usados, e suas principais vantagens em relação a satélites são a resolução de imagens e possibilidade de obter imagens a qualquer dia e horário~\cite{Disperati2007,Kestur2018}. Satélites são críticos em determinadas épocas do ano devido a presença de nuvens. 


Dada a disponibilidade de imagens de sensoriamento remoto, pesquisadores têm concentrado no desenvolvimento de técnicas que possam vir a substituir a análise humana das imagens captadas. Existe uma variedade de abordagens computacionais para a detecção de árvores, tais como casamento de \textit{template}, segmentação por crescimento de regiões, detecção de picos e métodos de aprendizado profundo. Métodos de casamento de \textit{template} constroem uma série de modelos para caracterizar os aspectos de árvores, levando em consideração a geometria da copa e propriedades radiométricas. Uma vez construídos os \textit{templates} um procedimento de janela deslizante é implementado para buscar os melhores casamentos, isto é, os locais de maiores probabilidades de existência de árvores. Métodos de segmentação por crescimento de regiões partem de vários pixels semente e vai agregando novos pixels na vizinhança conforme algum critério de similaridade. Métodos de filtragem máxima local são usados para detectar picos locais, dados por uma alta saturação de verde normalmente encontrada no interior das copas das árvores. Tal procedimento é aplicado por mecanismo de convolução usando uma janela deslizante de um tamanho específico dada pela discretização de uma função gaussiana. Métodos de aprendizagem profunda usando a abordagem convolucional têm sido empregados com sucesso em várias tarefas de visão computacional. Estes métodos aprendem uma extração de características através de uma representação interna de pesos da rede neural, que melhor classificam os dados. No Capítulo~\ref{chapter:correlatos} serão apresentadas, as principais pesquisas de cada uma das categoria citadas, para a detecção e contagem de árvores.


\section{Formulação do problema de pesquisa}

Dado uma imagem de satélite colorida (RGB) de alta resolução (de ao menos 2m/pixel) de uma área agrícola delimitada, nosso problema consiste em detectar e contar a quantidade de árvores utilizando técnicas automáticas de processamento de imagens. Outras possíveis informações que podem ser usadas para a resolução do problema, inclui o espaçamento entre árvores e estimativa da área ocupada pela copa de cada árvore. O resultado gerado pela resolução do problema incluirá a localização de cada árvore e/ou a delimitação da área ocupada por sua copa e a contagem das árvores. Pretende-se avaliar os resultados gerados gerados ao final deste trabalho com base em métricas tradicionais de detecção de objetos, tal como precisão, sensitividade, especificidade, entre outras  que são apresentadas no Capítulo~\ref{chapter:desenvolvimento}, em comparação com imagens de \textit{ground-truth} geradas por anotação manual.


\section{Objetivo}

Nesta pesquisa, técnicas de processamento digital de imagens serão aplicadas na área de agricultura, visando compor uma abordagem efetiva para detecção e contagem de árvores com base em imagens de satélite. Para se chegar a abordagem, pretende-se comparar técnicas alternativas para as subtarefas do processo, avaliando estas, através de métricas para mensurar a qualidade dos resultados obtidos.


\section{Metodologia}
As imagens serão coletadas pela API Python do Google Maps. Será aplicado o processamento que consiste de índices de vegetação, binarização, tratamento morfológico e aplicação da transformada da distância. Posteriormente, serão comparados  métodos clássicos de detecção de objetos. Até o momento experimentamos \textit{Watershed} e da filtragem máxima local. Por fim, será feita a rotulação dos objetos identificados, onde o número de rótulos serão o número de árvores estimadas.


\section{Principais resultados e contribuições iniciais}

Até o momento foram desenvolvidas duas metodologias para a contagens de árvores a partir de imagens de satélite de alta resolução. As metodologias consistem de um processamento comum à ambas: conversão das imagens para níveis de cinza, binarização, tratamento morfológico e aplicação da transformada da distância. Na segunda etapa do processo, comparamos segmentação por inundação (\textit{watershed}) com a abordagem de detecção de picos através do método de filtragem máxima local. Na terceira fase, é feita a rotulação dos objetos identificados, onde o número de rótulos corresponderá aos número de árvores estimado. Para PFC2 pretende-se comparar o desempenho de índices de vegetação, ajustar adequadamente os parâmetros envolvidos em cada metodologia, desenvolver uma base de \textit{ground-truth} e validar as metodologias com métricas objetivas de detecção de objetos, tais como sensitividade, especificidade, entre outras. 


\section{Obstáculos esperados e desafios iniciais}

%Conforme destacado na literatura, a contagem de árvores com base em imagens aéreas, coletadas por satélites ou veículos aéreos não-tripulados (VANTs), é importante sob vários aspectos. Por exemplo, \citeonline{Daliakopoulos2009} e \citeonline{Dorj2017} utilizaram a contagem de árvores com base em imagens para estimar o rendimento de culturas.
 
%\citeonline{Daliakopoulos2009} desenvolveu uma metodologia baseada em visão computacional, para monitorar o número de árvores com o passar do tempo. Os autores argumentam que essas contagens ao longo do tempo, pode resultar em índices importantes quando associados a doenças de árvores, condições climáticas e comportamento do ecossistema.

%Segundo \citeonline{Reis2007}, foi resolvido usar abordagens para contagem de árvores para resolver o problema de inventários florestais. Segundo os autores, os dados de inventários florestais têm sido coletados principalmente por pesquisas de campo, que são dispendiosas e demoradas, e muita das vezes com erros devido a falta de treinamento da equipe de campo.

Conforme reportado na literatura, apesar dos variados benefícios de análises, existem vários desafios para o desenvolvimento de métodos altamente precisos para a detecção e contagem de árvores. Entre estes desafios, estão a presença de nuvens que pode resultar em oclusões nas imagens capturadas, problemas envolvendo resolução de imagem e a dimensão da copa das árvores \cite{Srestasathiern2014}, além de problemas de sobreposição da copa das árvores \cite{Disperati2007}.

Segundo \cite{Daliman2016}, a análise de árvores individuais com base em imagens de sensoriamento remoto é um problema complexo, pois há variações do tamanho, da forma e da resposta espectral da copa das árvores. Com isso, o que é detectado como objetos únicos podem de fato corresponder à galhos da uma mesma árvore, ou a um grupo de árvores.

Outros desafios para o desenvolvimento de métodos de contagem de árvores, de acordo com o estudo de \cite{Daliman2016}, é que poderá haver erros de detecções de árvores, e isso ocorre por causa da proximidade de árvores vizinhas, árvores que encobrem outras árvores, árvores à sombra ou árvores que tem baixo contraste espectral com o fundo. Apesar desses problemas, é de extrema importância fazer a detecção devido as análises agrícolas e aplicações subsequentes.

No resultados iniciais pode-se perceber os problemas citados por \cite{Daliman2016}, principalmente devido ao fato de sobreposição da copa das árvores. Pretendemos mensurar quantitativamente e propor soluções para amenizar este problema na disciplina de PFC2 este será desenvolvidas imagens que \textit{ground-truth} que permitirá mensurar adequadamente a qualidade dos resultados obtidos.

\section{Organização da monografia}

O restante desse trabalho está dividido da seguinte forma. No Capítulo~\ref{chapter:conceitosPDI} são apresentados os principais conceitos básicos de processamento de imagens necessários para a compreensão das metodologias em desenvolvimento para a contagem de árvores. No Capítulo~\ref{chapter:correlatos} as pesquisas correlatas de contagens de plantas são categorizadas de acordo com as técnicas aplicadas e discutidas. No Capítulo~\ref{chapter:desenvolvimento} são apresentados o desenvolvimento e os resultados iniciais. Por fim, o Capítulo~\ref{chapter:conclusao} apresenta conclusões sobre o desenvolvimento feito na disciplina de PFC1 e o cronograma. 


\chapter{Conceitos de processamento de imagens para detecção e contagem de árvores}
\label{chapter:conceitosPDI}
\section{Considerações iniciais}
Neste capítulo serão descritos conceitos básicos de processamento de imagens usados na literatura para a detecção e contagem de árvores. Como essa pesquisa foca em abordagens baseada em detecção de picos e em segmentação por crescimento de regiões através do método baseado em inundação (\textit{watershed}). Os conceitos descritos são relacionados para o desenvolvimento de métodos baseados nessas abordagens. Para ambas as abordagens, está sendo desenvolvida uma etapa de pré-processamento comum, que tem como objetivo evidenciar as regiões das copas das árvores. Na seções a seguir, apresentamos primeiramente as técnicas que estão sendo investigadas para o pré-processamento, as quais são: índices de vegetação, binarização pelo método de Otsu, morfologia matemática e transformada da distância. Em seguida são apresentados os conceitos de segmentação por \textit{watershed} e de filtragem máxima local. Por fim, será apresentado conceitos de rotulação de componentes conectados e medida de desempenho de detecção que serão usados neste trabalho.   

\subsection{Índices de vegetação}

Basicamente um índice de vegetação~\cite{Torres2014}, é um cálculo sobre os valores espectrais de pixels, com o propósito de diferenciar os pixels de vegetação dos pixels de não-vegetação, tais como no solo, leitos de água, estradas, construções, entre outros. Um índice de vegetação, toma como entrada uma imagem colorida ou multiespectral\footnote{Uma imagem multiespectral consiste do imageamento de um mesmo objeto, tomado com base em diferentes comprimentos de ondas eletromagnéticas tais como luz visível, infravermelha, ultravioleta, raio-X ou qualquer outra faixa do espectro eletromagnético.}, e retorna uma imagem em níveis de cinza, onde os pixels que contém uma maior evidência de que seja vegetação, terá valores dissimilares dos pixels onde há mínima evidência que seja vegetação. \cite{Torres2014} revisou vários índices de vegetação para o mapeamento da fração de vegetação em plantio de trigo. Neste trabalho experimentaremos estes índices no contexto de detecção de copa de árvores através de imagens de satélite. 

A entrada para os cálculos dos índices de vegetação são os valores $R$, $G$ e $B$ dos sistema de cores RGB, a normalização destes dada pela Equação(~\ref{eq:normRGB}) ou combinações de índices.

\begin{equation}
    r = \frac{R}{R+G+B}; g = \frac{G}{R+G+B}; b = \frac{B}{R+G+B}
    \label{eq:normRGB}
\end{equation}

A seguir são listados os índices de vegetação apresentados em \cite{Torres2014}.

$\textbf{Índice de diferença verde-vermelho normalizado:}$
\begin{equation}
    NGRDI = \frac{G - R}{G + R}
\end{equation}

$\textbf{Excesso de verde:}$
\begin{equation}
    ExG(2) = 2g - r - b
\end{equation}

$\textbf{Índice de cores da vegetação:}$
\begin{equation}
    CIVE = 0.441r - 0.881g + 0.385b + 18.78745
\end{equation}

$\textbf{Vegetativen:}$
\begin{equation} 
    VEG = \frac{g}{r^{a} b^{(1-a)}}  \ \text{com} \ a = 0.667
\end{equation}

$\textbf{Excesso de verde menos excesso de vermelho:}$
\begin{equation}
    ExGR = ExG - ExR = ExG - 1.4r-g
\end{equation}

$\textbf{Índice Woebbecke:}$
\begin{equation}
    WI = \frac{g-b}{r-g}
\end{equation}

$\textbf{Combinação 1:}$
\begin{equation}
    COM(1) = 0.25*ExG + 0.3*ExGR + 0.33*CIVE + 0.12*VEG
\end{equation}

$\textbf{Combinação 2:}$
\begin{equation}
    COM(2) = 0.36*ExG + 0.47*CIVE + 0.17*VEG
\end{equation}

\subsection{Método de Otsu e binarização}

Segundo \cite{Gonzales2010}, o método de Otsu é ótimo no sentido de maximizar a variância entre classes. A ideia básica é separar os pixels em duas classes por um limiar apropriado, onde a variação de intensidade entre as classes seja máxima e a variação de intensidade dentro de cada classe seja mínima. O método de Otsu tem a propriedade importante de poder se basear inteiramente no histograma normalizado $p_i$, que dá a probabilidade de ocorrência de cada nível de cinza $i$ na imagem.

Seja $\{0, 1, 2, \ldots, L-1\}$, o conjunto de $L$ níveis de intensidade distintos em uma imagem digital de tamanho $M\times N$ pixels, e seja $n_{i}$, o número de pixels com intensidade $i$. O número total $MN$ de pixels na imagem é $MN = n_{0} + n_{1} + n_{2} + \ldots + n_{L-1} = M*N$. A probabilidade de ocorrência de cada pixel, também chamada de histograma normalizado, é dada  por $p_{i} = n_{i} / MN$, dos quais tem-se que:
\begin{equation}
    \sum_{i=0}^{L-1} p_{i}=1 \qquad \text{e} \qquad p_{i}\geq 0
\end{equation}

Agora, suponha selecionar um limite $T(k) = k, 0 < k < L-1$, e usá-lo para separar os pixels da imagem de entrada em duas classes, $c_{1}$ e $c_{2}$, onde $c_{1}$ consiste em todos os pixels na imagem com valores de intensidade no intervalo $[0,k]$ e $c_{2}$ consiste em pixels com valores no intervalo $[k+1, L-1]$. Usando essa limiarização, a probabilidade, $q_{1}(k)$, de um pixel petencer à classe $c_{1}$ é dada pela soma cumulativa:
\begin{equation}
    q_{1}(k) = \sum_{i=0}^{k} p_{i}
\end{equation}

Visto de outra maneira, $q_{1}(k)$ é a probabilidade da classe $c_{1}$ ocorrer. Da mesma forma, a probabilidade de ocorrência da classe $c_{2}$ é:
\begin{equation}
    q_{2}(k) = \sum_{i=k+1}^{L-1} p_{i} = 1- q_{1}(k)
\end{equation}

O valor médio da intensidade dos pixels em $c_{1}$ é:
\begin{equation}
    \begin{split}
    m_{1}(k) = \sum_{i=0}^{k} \frac{ip_{i}}{q_1(k)}
    \end{split}
\end{equation}

Da mesma forma, o valor médio da intensidade dos pixels atribuídos à classe $c_{2}$ é: 
\begin{equation}
    \begin{split}
        m_{2}(k) = \sum_{i=k+1}^{L-1} \frac{ip_{i}}{q_2(k)}
    \end{split}
\end{equation}

A variância dentro da classe $c_1$, em que os valores de pixels estão no intervalo $[0,k]$, é dada por:

\begin{equation}
    \sigma^2_1(k) =  \sum_{i=0}^{k}[i-m_1(k)] \frac{p_i}{q_1(k)} 
\end{equation}

De forma similar, a variância dentro da classe $c_2$, em que os valores de pixels estão no intervalo $[k+1,L-1]$, é dada por:

\begin{equation}
    \sigma^2_1(k) =  \sum_{i=k+1}^{L-1}[i-m_2(k)] \frac{p_i}{q_2(k)} 
\end{equation}

Para encontrar o melhor valor de $k$, denominado de $k^{*}$, simplesmente avaliamos a Equação~\ref{eq:var_inter} para todos os valores inteiros de $k, 0 < k < L-1$ e tomamos o valor de $k$ que produz o maior valor de variância inter classe $\sigma^2(k)$; 

\begin{equation}
     \argmax_k \sigma^2(k) = q_1(k)\sigma^2_1(k) + q_2(k)\sigma^2_2(k)
     \label{eq:var_inter}
\end{equation}

Caso exista mais de um valor de $k$ que produza o valor máximo de $\sigma^{2}$, é habitual calcular a média destes.

Uma vez obtido $k^{*}$, a imagem de entrada $f(x, y)$ é limiarizada da seguinte forma:
\begin{equation}
      g(x,y) = \left \{  \begin{array}{cc}
        1,  & \text{se} \ f(x,y) > k^{*}\\
        0,  & \text{se} \ f(x,y) \leq  k^{*} \\
    \end{array} \right \}
\end{equation}
para $x = 0,1,2,\ldots,M-1$ e $y = 0,1,2,\ldots,N-1$. Observe que todas as quantidades necessárias para avaliar a Equação~\ref{eq:var_inter} são obtidas usando apenas o histograma de $f(x,y)$. Além do limiar ideal, outras informações sobre a imagem segmentada podem ser extraídas do histograma. Por exemplo, $q_{1}$($k^{*}$) e $q_{2}$($k^{*}$), indicam as probabilidades de cada classe conforme o limiar $k^*$. Da mesma forma, as médias $m_{1} (k^{*})$ e $m_{2} (k^{*})$ são estimativas da intensidade média das classes na imagem original. A Figura~\ref{fig:Otsu} ilustra a aplicação do método de Otsu. Em (a) é dada a imagem em níveis de cinza sobre a qual é aplicada o método; em (c) é mostrado histograma da imagem (a) juntamente com o \textit{threshold} encontrado pelo método de Otsu, denotado pela linha tracejada. Em (b) é dada a imagem binarizada resultante da aplicação do \textit{threshold}. Neste caso foi considerado que os valores abaixo do \textit{threshold} recebem o valor 1 e os valores acima ou igual ao \textit{threshold} recebem o valor 0.

 
\begin{figure}[h]
\center
\caption{Ilustração da aplicação do método de Otsu.}
\subfigure[TCC UFG/images][Imagem em nível de cinza]{\includegraphics[width=7cm]{mangaGray.png}} \ \ \ 
\subfigure[TCC UFG/images][Imagem binarizada]{\includegraphics[width=7cm]{mangaBinaria.png}} \ \ \
\subfigure[TCC UFG/images][Threshold da imagem]{\includegraphics[width=7.5cm]{mangaThreshold.png}}
\label{fig:Otsu}

Fonte: Elaborada pela autora
\end{figure}


\subsection{Morfologia Matemática para imagens binárias}

Na área de Biologia a palavra `morfologia' geralmente lida com a forma e a estrutura de animais e plantas. Em processamento de imagens, morfologia matemática constitui ferramentas para extrair componentes de imagem que são úteis na representação e descrição da forma da região. A base para a descrição das operações morfológicas é a  teoria dos conjuntos. Este trabalho descreve os conceitos básicos de morfologia matemática para imagens binárias tomando como base o livro de \cite{Gonzales2010}.

Em imagens binárias especialmente, seus conjuntos contém elementos do espaço inteiro $\mathbb{Z}^{2}$, em que cada elemento de um conjunto é uma tupla (vetor 2-D), cujas coordenadas são de um pixel de objeto na imagem. 

A morfologia para imagens binárias contém dois tipos de conjuntos de pixels: objetos ($A$) e elementos estruturantes ($B$). Os objetos são conjuntos de pixels dos objetos de interesse, já os elementos estruturantes são especificados para fazer análise e processamentos dos objetos de interesse, e normalmente levam em conta as características geométricas destes. 

\subsubsection{Erosão}
Sendo $A$ e $B$ conjuntos em $\mathbb{Z}^{2}$, a erosão de $A$ por $B$, denota $A \ominus B$, é definida como:
\begin{equation}
     A\ominus B = \left\{z|(B)_{z} \subseteq A\right\}
     \label{eq:erosao}
\end{equation}
onde $A$ é um conjunto de pixels de objetos, $B$ é um elemento estruturante e $z$ são translações (deslocamentos) de $B$ ao longo do plano da imagem. Em outras palavras, a Equação~\ref{eq:erosao} indica que a erosão de $A$ por $B$ é o conjunto de todos os pontos $z$, de modo que $B$, transladado por $z$, está contido em $A$. Na prática, a erosão elimina partes nas extremidades dos objetos, podendo servir ao propósito de separação de objeto conectados. Contudo, ela diminui o tamanho dos objetos. 

A figura \ref{fig:erosao} ilustra a aplicação da operação de erosão. As imagens usadas foram extraídas de nossa base de imagens. Em (a) é dada a imagem binarizada; Em (b) é dada a imagem resultante da aplicação da operação de erosão.  O elemento estruturante usado é um retângulo de dimensão $3 \times 3$ para a operação de erosão. 

\begin{figure}[h]
\center
\caption{Ilustração da aplicação da operação de erosão.}
\subfigure[TCC UFG/images][Imagem binarizada]{\includegraphics[width=7cm]{mangaBinaria.png}}   \ \ \
\subfigure[TCC UFG/images][Imagem erodida]{\includegraphics[width=7cm]{mangaErodida.png}}
\qquad
\label{fig:erosao}
Fonte: Elaborada pela autora
\end{figure}

\subsubsection{Dilatação}
Novamente sendo $A$ e $B$ conjuntos em $\mathbb{Z}^{2}$, a dilatação de $A$ por $B$, denotada como $A \oplus B$, é definida como: 
\begin{equation}
    A \oplus B = \left \{ z | (\hat{B})_{z}\cap A \neq \varnothing \right \}
\end{equation}
sendo $\hat B $ a reflexão de $B$. Em resumo, a dilatação de $A$ por $B$ é o conjunto de todos os deslocamentos $z$, de modo que os elementos de $\hat B$ se sobrepõem a pelo menos um elemento de $A$. Na prática a dilação expande as fronteiras dos objetos, podendo servir para ligar partes desconexas de um mesmo objeto. Contudo, ela aumenta o tamanho dos objetos.

A figura \ref{fig:dilatacao} ilustra a aplicação da operação de dilatação. As imagens usadas foram extraídas de nossa base de imagens. Em (a) é dada a imagem binarizada; Em (b) é dada a imagem resultante da aplicação da operação de dilatação.  O elemento estruturante usado é um retângulo de dimensão $3 \times 3$ para a operação de dilatação.

%FIGURA DILATAÇÃO
\begin{figure}[h]
\center
\caption{Ilustração da aplicação da operação de dilatação.}
\subfigure[TCC UFG/images][Imagem binarizada]{\includegraphics[width=7cm]{mangaBinaria.png}}   \ \ \
\subfigure[TCC UFG/images][Imagem dilatada]{\includegraphics[width=7cm]{mangaDilatada.png}}
\label{fig:dilatacao}

Fonte: Elaborada pela autora
\end{figure}

\subsubsection{Abertura}

A abertura do conjunto $A$ pelo elemento estruturante $B$, indicado por $A \circ  B$, é definida como:
\begin{equation}
    A \circ B = \left ( A \ominus  B \right )\oplus  B
\end{equation}

Assim, a abertura $A$ por $B$ é a erosão de $A$ por $B$, seguida de uma dilatação do resultado por $B$. 
A abertura geralmente suaviza o contorno de um objeto, quebra os istmos estreitos e elimina saliências finas.

A figura \ref{fig:abertura} ilustra a aplicação da operação de abertura. As imagens usadas foram extraídas de nossa base de imagens. Em (a) é dada a imagem binarizada; Em (b) é dada a imagem resultante da aplicação da operação de abertura.  O elemento estruturante usado é um retângulo de dimensão $3 \times 3$ para a operação de abertura.

%FIGURA ABERTURA
\begin{figure}[h]
\center
\caption{Ilustração da aplicação da operação de abertura.}
\subfigure[TCC UFG/images][Imagem binarizada]{\includegraphics[width=7cm]{mangaBinaria.png}}   \ \ \
\subfigure[TCC UFG/images][Imagem com abertura]{\includegraphics[width=7cm]{mangaAbertura.png}}
\label{fig:abertura}

Fonte: Elaborada pela autora
\end{figure}

\subsubsection{Fechamento}
O fechamento do conjunto $A$ pelo elemento estruturante $B$, denominado $A \bullet B$, é definido como:
\begin{equation}
    A \bullet  B = \left ( A \oplus  B \right )\ominus  B
\end{equation}
ou seja, o fechamento de $A$ por $B$ é dado pela dilatação de $A$ por $B$, seguido pela erosão do resultado por $B$. 
O fechamento tende a suavizar seções de contornos, mas, ao contrário da abertura, geralmente funde quebras estreitas e golfos finos e longos, elimina pequenos orifícios e preenche lacunas no contorno. 

A figura \ref{fig:fechamento} ilustra a aplicação da operação de fechamento. As imagens usadas foram extraídas de nossa base de imagens. Em (a) é dada a imagem binarizada; Em (b) é dada a imagem resultante da aplicação da operação de fechamento. O elemento estruturante usado é um retângulo de dimensão $3 \times 3$ para a operação de fechamento. 

%FIGURA FECHAMENTO
\begin{figure}[h]
\center
\caption{Ilustração da aplicação da operação de fechamento.}
\subfigure[TCC UFG/images][Imagem binarizada]{\includegraphics[width=7cm]{mangaBinaria.png}}   \ \ \
\subfigure[TCC UFG/images][Imagem com fechamento]{\includegraphics[width=7cm]{mangaFechamento.png}}
\label{fig:fechamento}

Fonte: Elaborada pela autora
\end{figure}

\subsection{Transformada de Distância}

A transformada da distância~\cite{Fabbri2008} mede a distância de cada ponto de objeto até a borda mais próxima. Seja uma imagem bidimensional $I$ consistindo de duas classes de pixels: pixels de objeto (de valor 1) e pixels de não objeto (de valor 0).  Basicamente a transformada de distância gera uma matriz da mesma dimensão da imagem original binária, onde para cada pixel de objeto é atribuído a distância deste pixel para o pixel de borda mais próximo. Esta matriz, quando visualizada como uma imagem em níveis de cinza, produzirá valores mais claros para pixels no centro de um objeto e valores mais escuros para pixels perto das borda do objeto. Normalmente se utiliza uma métrica como distância.


Uma métrica bastante conhecida e usada em processamento de imagens denominada $L_{p}$ \cite{Maurer2003}, que é dada pela seguinte Equação:
\begin{equation}
    \Delta (P,Q) = \left ( \sum_{i=1}^{k}\mid P_{i} - Q_{i} \mid ^p \right )\tfrac{1}{p} \ \ ,
\end{equation}
onde $P$ e $Q$ são $k$-tuplas, $P_{i}$ e $Q_{i}$  são as $i$-ésimas coordenadas de $P$ e $Q$, e $1 \leq p \leq \infty$. No caso de imagens bidimensionais o número de coordenadas $k$ é igual a 2. As métricas $L_{1}$, $L_{2}$ e $L_{\infty}$ são conhecidas como distâncias de Manhattan ou quarteirão, Euclidiana e de tabuleiro de xadrez. 

Uma métrica $\Delta$: $\mathbb{R}^{k} \times \mathbb{R}^{k} \rightarrow \mathbb{R}$ satisfaz as seguintes propriedades:

$\textbf{Propriedade 1}$: Positividade\\  \indent \indent $\Delta (P,Q) = 0 \ \  \text{se} \ \ P = Q$

$\textbf{Propriedade 2}$: Simetria\\ 
\indent \indent $\Delta(P,Q) = \Delta(Q,P) \ \text{para quaisquer} \ P \ \text{e} \ Q $

$\textbf{Propriedade 3}$: Desigualdade triangular\\
\indent \indent $\Delta(P,S)\leq  \Delta(P,Q) + \Delta(Q,S) \ \text{para quaisquer} \ P , \ Q, \ and \ S $

$\textbf{Propriedade 4}$: Monotonicidade\\ 
\indent \indent Seja $P$ e $Q$ duas $k$-tuplas que diferem apenas nos valores das $d$-ésimas coordenadas (ou seja, $P_{i}$ = $Q_{i}$, $i \neq d$). Para concretude, assuma que $P_{d}$ < $Q_{d}$. Para qualquer $U$ e $V$ de tal forma que 1) $\Delta$(P,$U$) $\leq$ $\Delta$(P,V) e $\Delta$(Q,V) < $\Delta$(Q,$U$) ou 2) $\Delta$(P,$U$) < $\Delta$(P,V) e $\Delta$(Q,V) $\leq$ $\Delta$(Q,$U$) tem-se que $U_{d} < U_{d}$.

$\textbf{Propriedade 5}$\\
\indent \indent Seja $U$ e $V$ duas $k$-tuplas com valores que diferem apenas nas  $d$-ésimas coordenadas (ou seja, $U_{d}$ = $V_{d}$. Seja $U$ e $V$ $k$-tuplas com valores idênticos da $d$-ésimas coordenas (isto é, $U_d = V_d$). Se $\Delta(P,U) \leq \Delta(P, V)$, \text{então } $\Delta(Q,U) \leq \Delta(Q, V)$.

A figura \ref{fig:transDistancia} ilustra a transformada da distância. As imagens usadas foram extraídas de nossa base de imagens. Em (a) é dada a imagem binarizada; Em (b) é dada a imagem resultante da aplicação da transformada da distância.  O elemento estruturante usado é um retângulo de dimensão $3 \times 3$ para a transformada da distância. 

%FIGURA TRANSFORMADA DA DISTANCIA
\begin{figure}[h]
\center
\caption{Ilustração da transformada da distância.}
\subfigure[TCC UFG/images][Imagem binarizada]{\includegraphics[width=7cm]{mangaBinaria.png}}   \ \ \
\subfigure[TCC UFG/images][Imagem com Transformada da distância]{\includegraphics[width=7cm]{mangaTransDistancia.png}}
\label{fig:transDistancia}

Fonte: Elaborada pela autora
\end{figure}

\subsection{Watershed}
Segundo \cite{Gonzales2010}, o conceito de \textit{watershed} (bacia hidrográfica) é baseado na visualização de uma imagem em três dimensões; duas coordenadas espaciais \textit{versus} intensidade. Nessa interpretação ``topográfica'', consideramos três tipos de pontos: 1) pontos pertencentes a um mínimo regional; 2) pontos nos quais uma gota de água, se colocada no local de qualquer um desses pontos, cairia com certeza em um mínimo único; e 3) pontos em que a água teria a mesma probabilidade de cair para mais de um mínimo. Para um mínimo regional específico, o conjunto de pontos que satisfazem a condição 2 é chamado de bacia hidrográfica ou bacia hidrográfica desse mínimo. Os pontos que satisfazem a condição 3, formam linhas de crista na superfície topográfica e são chamados de linhas de divisão ou linhas de bacias hidrográficas.


O principal objetivo desse algoritmo, é encontrar as linhas da bacia hidrográfica. Uma das principais aplicações, é a extração de objetos de tonalidades quase uniformes. Na prática, geralmente, a segmentação de bacias hidrográficas é aplicada ao gradiente de uma imagem, e não a própria imagem. Regiões caracterizadas por pequenas variações de intensidade têm pequenos valores de gradiente. Assim, nesta formulação, os mínimos regionais das bacias hidrográficas se correlacionam muito bem com o pequeno valor do gradiente correspondente aos objetos de interesse.

A construção da barragem do algoritmo de segmentação das bacias hidrográficas, é baseada em imagens binárias, que são membros do espaço 2-D de inteiro $\mathbb{Z}^{2}$. A maneira mais simples de construir barragens que separam conjuntos de pontos binários, é usar a dilatação morfológica.
Primeiramente, para a construção das barragens é aplicado a dilatação que simula a inundação. Se a água derrama de uma bacia para outra, uma barragem deve ser construída para impedir que isso aconteça. Seja $M_{1}$ e $M_{2}$  conjuntos de coordenadas de pontos em dois mínimos regionais. Seja $C_{n-1} (M_{1})$ e $C_{n-1} (M_{2})$ os conjuntos de coordenadas de pontos na bacia hidrográfica associado a esses dois mínimos no estágio $n-1$ da inundação.

Dois componentes conectados que se tornaram um único componente, indicam que a água entre as duas bacias hidrográficas, se fundiu na etapa de inundação $n$. Esse componente fundido é indicado por $q$. 

Suponha que cada um dos componentes conectados seja dilatado pelo elemento estruturante, sujeito a duas condições: 1) A dilatação deve ser restringida a $q$ (isso significa que o centro do elemento estruturante pode ser localizado apenas nos pontos em $q$ durante a dilatação); e 2) A dilatação não pode ser realizada em pontos que causariam a união dos conjuntos.

 É evidente que os únicos pontos em $q$, que satisfazem as duas condições, descrevem o caminho conectado de um pixel com hachuras cruzadas. Esse caminho é a barragem de separação desejada no estágio $n$ das inundações. A construção da barragem nesse nível de inundação, é concluída definindo todos os pontos no caminho apenas determinado para um valor maior que o valor máximo de intensidade possível da imagem (por exemplo, maior que 255 para uma imagem de 8 bits). Isso impedirá que a água atravesse a parte da barragem concluída à medida que o nível de inundação aumenta. 

A figura \ref{fig:conceitos} ilustra o \textit{Watershed}. Primeiramente as bacias hidrográficas são parcialmente inundadas no estágio n-1 da inundação; As inundações no estágio n mostram que a água derramou entre bacias; É mostrado o elemento estruturante usado para dilatação; Resultado da dilatação e construção da barragem.

%FIGURA WATERSHED
\begin{figure}[ht]
\centering
\caption{Ilustração do \textit{Watershed}.}
\includegraphics[width=8cm, height=10cm]{images/watershed.png}

Fonte: \citeonline{Gonzales2010}
\label{fig:conceitos}
\end{figure}

\subsection{Filtragem máxima local}

Basicamente o método de filtragem máxima local aplica na imagem uma convolução por um \textit{kernel} de suavização bidimensional em formato de `chapéu mexicano'. Após a convolução procura-se pelos pontos máximos na imagem resultante. Tradicionalmente, um dos trabalhos pioneiros de filtragem máxima local~\cite{dralle1996}, que inicialmente denominou o método de suavização por \textit{kernel}, o \textit{kernel} é dado por uma função gassiana bidimensional. Em \cite{dralle1996} este foi definido com um kernel gaussiano arbitrário determinado pelos parâmetros $\sigma_{1}$, $\sigma_{2},$ e  $\rho$. Seja $x_{i} = (x_{i1},x_{i2})^{T}$ onde \textit{i=1,2, ..., I}, denotam pixels centrais dos pixels \textit{I} de uma imagem. Seja:
 \begin{equation}
     \sum  = \binom{\sigma_{1}^{2} \ \ \ \ \ \ \ \ \rho\sigma_{1}\sigma_{2}}{\rho\sigma_{1}\sigma_{2} \ \ \ \ \ \ \ \ \sigma_{2}^{2}} 
 \end{equation}
 e
 
 \begin{equation}
     G_{ij} = k_{i} \ exp \left ( -\frac{1}{2}(x_{i} - x_{j})^T \sum ^{-1}(x_{i} - x_{j})  \right )
 \end{equation}
onde \textit{i = 1,2, ..., I}, \textit{j = 1,2, ..., I} e $k_{i}$ é uma constante de normalização tal que:
\begin{equation}
    \sum_{j=1}^{I} G{ij}= 1, \ \ \ \ \ \ \ \ i = 1,2,..., I
\end{equation}
 Foi notado que $k_{i}$ é essencialmente constante para pixels que não estão próximos da borda da imagem.
 
 Seja $V_i$, \textit{i=1,2, ..., I} os valores de nível de cinza dos pixels em uma imagem. Em seguida, os níveis de cinza da imagem suavizada do kernel são fornecidos pela operação de filtragem:
 \begin{equation}
     V'_{i} = \sum_{j=1}^{I}G_{ij}V_{j}, \ \ \ \ i=1,2, ...,I
 \end{equation}
 
Em \cite{dralle1996} é calculado o histograma de níveis de cinza da imagem filtrada,e é considerado como árvores picos cujo valor é maior ou igual ao nível cinza correspondente à moda do histograma, isto é, o nível de cinza mais comum na imagem (nível de cinza de frequência máxima no histograma). Picos abaixo desse valor não são considerados como árvores, mas indicam uma variação na intensidade de luz.
 
\subsection{Rotulação de Componentes conectados}

Segundo \cite{Gonzales2010}, a capacidade de extrair componentes conectados de uma imagem binária é central para muitas aplicações automatizadas de análise de imagem. Seja $A$ um conjunto de pixels de objetos que consiste em um ou mais componentes conectados. Seja $X_{0}$ um pixel de objeto de $A$. Pode-se extrair o componente conectado pelo procedimento interativo dado por: 
\begin{equation}
    X_{k} = (X_{k-1} \oplus B) \cap I \qquad \qquad k = 1,2,3...
\end{equation}
onde $B$ é o elemento estruturante que define a conectividade. O procedimento termina quando $X_{k} = X_{k-1}$. Nesse momento pode-se extrair o componente de A ou rotular este. Este procedimento deve ser repetido para um novo pixel de objeto não rotulado, até que todos os componentes conectados sejam extraídos ou rotulados.

A figura \ref{fig:rotulacao} ilustra a rotulação de componentes conectados. As imagens usadas foram extraídas de nossa base de imagens. Em (a) é dada a imagem binarizada; Em (b) é dada a imagem resultante da aplicação da rotulação de componentes conectados. O elemento estruturante usado é um retângulo de dimensão $3 \times 3$ para a rotulação de componentes conectados.

%FIGURAS: Binária e rotulada
\begin{figure}[h]
\center
\caption{Ilustração da rotulação de componentes conectados.}
\subfigure[TCC UFG/images][Imagem binarizada]{\includegraphics[width=7cm]{mangaBinaria.png}}   \ \ \
\subfigure[TCC UFG/images][Imagem rotulada]{\includegraphics[width=7cm]{mangaRotulada.png}}
\label{fig:rotulacao}

Fonte: Elaborada pela autora
\end{figure}

\subsection{Medidas de desempenho de detecção}

Conforme \cite{Szeliski2010}, pode-se quantificar o desempenho de um algoritmo de detecção de objetos, contando  o número de correspondências verdadeiras e falsas, e falhas de correspondência, usando as seguintes definições:

\textbf{TP:} verdadeiros positivos, isto é, a quantidade de árvores que foram preditas como árvores;

\textbf{FN:} falsos negativos, a quantidade de árvores que não foram detectadas;

\textbf{FP:} falsos positivos, a quantidade de objetos preditos como árvores, que não são árvores;

\textbf{TN:} negativos verdadeiros, a quantidade de objetos que não são árvores, e que não foram identificados como tal.


Através das medidas elementares anteriores, pode-se compor várias medidas, tais como:

\noindent \textbf{Taxa de verdadeiros positivos (\textit{true positive rate} -- TPR)}, também conhecida como sensitividade ou revocação:
\begin{equation}
    TPR: \frac{TP}{TP + FN} = \frac{TP}{P}
\end{equation}
mede a taxa dos exemplos positivos que foram preditos corretamente, ou seja, a taxa de árvores que foram detectadas.\\

\noindent \textbf{Taxa de falso positivos (\textit{false positive rate} -- FPR)}, também conhecida como taxa de falso alarme:
\begin{equation}
    FPR: \frac{FP}{FP+TN} = \frac{FP}{N}
\end{equation}
mede a taxa de objetos preditos erroneamente como árvores.\\

\noindent \textbf{Valor preditivo positivo (\textit{positive predictive value} -- PPV),} também chamado de precisão (\textit{precision}):,
\begin{equation}
    PPV = \frac{TP}{TP+FP}
\end{equation}
mede a probabilidade de um objeto identificado como árvore, ser de fato uma árvore.\\

\noindent {\textbf{Acurácia (\textit{accuracy} -- ACC}:}
\begin{equation}
    ACC = \frac{TP+TN}{P+N};
\end{equation}
mede a taxa de acerto da classificação entre árvore e não árvore.

\section{Considerações finais}

Este capítulo abordou os conceitos principais de processamento de imagens que serão utilizados no desenvolvimento de metodologias para a detecção e contagem de árvores com base em imagens de satélite, os quais são: índices de vegetação, binarização, morfologia matemática, transformada da distância, segmentação baseada no conceito de inundação (\textit{watershed}), filtragem máxima local e rotulação. Por fim apresentamos medidas de desempenho de detecção que serão usadas para mensurar a qualidade dos resultados de identificação e contagem de árvores. No próximo capítulo são categorizadas e descritas as principais pesquisas de detecção e contagem de plantas da literatura.

\chapter{Trabalhos correlatos}
\label{chapter:correlatos}
\section{Considerações iniciais}


Este capítulo explora metodologias existentes na literatura que são capazes de realizar detecção e contagem de plantas. Primeiramente será descrito o tipo de imagem pesquisada, em seguida a espécie da planta, as técnicas usadas e a sua taxa de precisão.


%Nome autores: Vibha L, P Deepa Shenoy, Venugopal K R, L M Patnaik
%Nome do artigo: Using Cloud Bursting to Count Trees and Shrubs in Sub-Saharan Africa
Segundo \citeonline{Vibha2009}, o tipo de imagens coletadas e estudadas por eles, foram imagens de sensoriamento remoto do satélite QuickBird. O objetivo deste trabalho, foi criar uma abordagem para a segmentação da imagem e contar árvores. Segundo os autores, fazer contagem manualmente em terrenos florestais é uma tarefa maçante, e com isso, teve motivação de desenvolver este projeto para potencializar o processo de contagem. Por suas imagens estarem com uma qualidade muito ruim, foi aplicada a técnica de filtragem média no pré-processamento. Na etapa de contagem de árvores teve duas fases: Na primeira, faz o aprimoramento usando limiarização automática e na segunda fase,foi criado um modelo para a árvore chamado matriz de imagem, juntamente com uma função de mapeamento. Eles fizeram seus programas em Matlab 7 e a taxa de precisão média da contagem foi de 88\%. 


%Nome autores: Attilio Antonio Disperati, João Roberto dos Santos, Paulo Costa de Oliveira Filho e Till Neeff 
%Nome titulo: Aplicação da técnica “filtragem de locais máximas” em fotografia aérea digital para a contagem de copas em reflorestamento de Pinus elliottii
\citeonline{Disperati2007}, o tipo de imagens coletadas para serem estudadas, foram imagens de sensoriamento remoto. O objetivo deles era contar o topo das árvores de Pinus elliottii. A área das plantações das imagens que foram desenvolvidas, está na Floresta Nacional de Irati que situa-se no Paraná. Aplicaram a técnica de filtragem local máxima. O principal software usado por eles foi feito em C++. Ao usarem imagens de 600 dpi, tiveram 97\% de precisão na identificação das copas das árvores. Nas imagens com 100, 200 e 300 dpi foram aplicados quatro tamanhos de filtros combinados com a estatura das copas das imagens. Na pesquisa dos autores, foi comprovado que quanto menor a estatura do filtro, maior foi o número de pontos de máximas encontrados. Ao usar imagem com resolução de 300 dpi, tiveram 70,7\% de precisão.


%Titulo: Tree Crown Detection, Delineation and Counting in UAV Remote Sensed Images: A Neural Network Based Spectral–Spatial Method
%Autores: Ramesh Kestur , Akanksha Angural, Bazila Bashir, S. N. Omkar , M. B. Meenavathi , Gautham Anand 
Segundo \citeonline{Kestur2018}, as imagens estudadas por eles foram feitas por meio de dois UAVs de características distintas, porém os dois produzem imagens de alta resolução espacial. O objetivo deste trabalho, foi classificar a espectro-espacial dessas imagens RGB de alta resolução espacial para detectar e contar árvores. O tipo de plantas estudadas por eles foram bananeiras, mangueiras e conqueiros. Foi feito classificação supervisionada usando ELM(aprendizado de máquina extremo). O ELM foi modelado para valores RGB como vetores de recursos de entrada e classe de saída binária. Posteriormente, foi aplicado classificação espacial usando técnica de filtragem de propriedades geométricas com limiares.  Fazeram segmentação da imagem usando Watershed aplicando o método de Transformação e Distância. Por fim, comparam o método ELM com o método espectro-espacial do k-means. O método ELM teve maior acurária.


%Improving the Precision of Tree Counting by Combining Tree Detection with crown Delineation and classification on Homogeneity Guided Smoothed High Resolution(50 cm) Multispectral airbone Digital Data
Segundo \citeonline{Katoh2012} usaram imagens de  dados de sensoriamento remoto.O local estudado se localiza em uma floresta do campus da Universidade de Shinshu-Japão. Fizeram estudo de coníferas de várias espécies, tais como: C.obtusa, C.pisifera, P. densiflora, L.kaempferi, C. japonica e árvores com as folhas de Carvalho. O foco deste trabalho foi detectar as espécies e fazer a contagem delas. O processo seguido no trabalho deles foi: 1- separar as regiões do tipo cobertura homogênea(separadas por textura ou níveis de cinza); 2- Aplicou-se a técnica de filtragem máxima local; 3- Thresholding; 4- Usou-se classificação supervisionada com uma regra de decisão de máxima verossimilhança, para comparar as espécies de árvores. Por fim, a precisão do número total de árvores contadas por espécie, foi superior a 84\%.


%Titulo: Oil Palm Tree Detection with High Resolution Multi-Spectral Satellite Imagery
%Autores: Panu Srestasathiern  and Preesan Rakwatin
No trabalho de \citeonline{Sresta2014}, foi estudado imagens de satélites de alta resolução espacial, e o foco foi contar plantas de dendezeiros. O processo feito para fazer a contagem é começar aplicando o índice de vegetação, já para melhorar a separabilidade entre as copas dos dendezeiros e o fundo da imagem, foi aplicado a transformação de classificação. Posteriormente, foi aplicado o algoritmo de supressão não máximo, e por fim, foi aplicada a análise semi-variograma. Neste trabalho, a contagem teve 90\% de acurácia.


%Titulo: Deep Learning Based Oil Palm Tree Detection and Counting for High-Resolution Remote Sensing Images
%Autores: Weijia Li ; Haohuan Fu; Le  and Arthur Cracknell 
Segundo \citeonline{Li2017}, foram estudadas imagens de sensoriamento remoto de alta resolução em seu trabalho. O foco foi detectar e contar dendezeiros. As características das árvores estudadas por eles são lotadas e tem sobreposições. O processo que eles seguiram, foi primeiramente treinar uma rede neural convolucional, fez ajustes nela e posteriormente, preveram os rótulos de todas as bases de imagens que são coletadas pela técnica da janela deslizante. Usaram o framework Tensorflow e ainda compararam com 3 outros métodos. O método deles teve o melhor resultado, obtendo 96\% de precisão.


%Titulo: Comparing Boosted Cascades to Deep Learning Architectures for Fast and Robust Coconut Tree Detection in Aerial Images
%Autores: Steven Puttemans , Kristof Van Beeck  and Toon Goedemé
Segundo \citeonline{Puttemans2018}, as imagens usadas por eles são através de imagens aéreas e a planta específica que foi explorada são coqueiros. O objetivo deles foi fazer deteção e classificação usando dois tipos diferentes de abordagens e depois compará-las. As técnicas que os autores estudaram foi a aplicação de aprendizado profundo e cascatas(técnicas que foram exploradas em seus respectivos trabalhos correlatos). Usaram 3 estruturas disponíveis para o treinamento: 1- Cascata aumentada usando OPENCV; 2- Cascata reforçada usando MATLAB. Para desenvolver o modelo de deep learning, usaram C e CUDA. Usaram o framework Darknet para classificação e o YOLOv2 para detecção. A taxa de precisão da técnica da cascata de melhor desempenho foi de 94,56\%, enquanto o modelo de aprendizado profundo tem 97,4\% de acurácia.


%Detection of Individual Tree Crowns in Airborne Lidar Data – Koch 2006
%(Detecção de copas de árvores individuais em dados aéreos de Lidar )
Segundo \citeonline{koch2006}, as imagens estudadas em seu trabalho foram capturadas através da base de dados do Lidar aéreo. O foco deste artigo foi efetuar detecção de copas e altura das árvores, e o tipo delas é árvores coníferas. O estudo delas foi feito para saber se a estrutura delas é adequada,foi mencionado que essas informações são importantes na área de Engenharia Florestal. As implementações foram feitas na linguagem C++. Eles usaram suavização gaussiana para separar a diferença do tamanho das árvores. Já para detectar a coroa das árvores, foi usado o filtro máximo local. Por fim, 87,3\% das árvores são detectadas corretamente.


%Comparison of airborne and satellite high spatial resolution data for the identification of individual trees with local maxima filtering – Wulder 2004
%(Comparação de dados de alta resolução espacial no ar e por satélite para a identificação de árvores individuais com filtragem máxima local)
Segundo \citeonline{wulder2004}, as imagens estudadas foram da base de imagens de satelite IKONOS, por elas terem uma boa resolução.  Os tipos de árvores estudados foram Abeto-de-Douglas e Centro Vermelho Ocidental, que são plantas típicas de florestas.  Os autores também fazem comparações de imagens entre os satélites MEISII e IKONOS. A técnica usada para a detecção das árvores foi a de filtragem máxima local. O IKONOS teve 85\% de precisão e 51\% de erros de comissão, já o MEISII tem precisão de 67\% e 22\% de erro de comissão.


%Individual tree detection on variable and fixed Window size local maximum filtering applied to IKONOS imagery for enven-aged Eucalyptus plantation forests – GEBRESLASIE 2011
Segundo o autor \citeonline{Gebreslasie2011}, o  tipo de imagem estudado por ele, é por via de imagens de sensoriamento remoto por satélite. O tipo de árvore estudado por eles é plantações de Eucalypto de mesmo tamanho. As técnicas usadas por eles foi suavização gaussiana e classificação de imagem usando a abordagem de filtragem de máxima local. A taxa de precisão foi de 85\% de acurácia. 


%Automated extraction of tree and plot-based parameters in Citrus orchards from aerial images
%(Extração automatizada de parâmetros de árvores e parcelas em pomares de citros a partir de imagens aéreas) – Recio 2013
Segundo \citeonline{recio2013}, o tipo de imagens estudadas são a partir de imagens aéreas de alta resolução espacial, a área de estudo se localiza em Valência-Espanha. O objetivo foi extrair fração da cobertura arbórea, número de árvores e padrões de plantio. O tipo de planta estudado é Citros. O processo de desenvolvimento deles possui as seguintes etapas: Aplicação de uma classificação não-superviosionada com o algoritmo k-means, seguida pela identificação automática das classes que representam as árvores. Após este procedimento, a árvore é individualizada usando morfologia na imagem binarizada. A taxa de precisão na detecção foi superior a 80\%.


%A tree counting algorithm for precision agriculture tasks
%(Um algoritmo de contagem de árvores para tarefas de agricultura de precisão) – Santoro 2013
Segundo o autor \citeonline{franco2013}, os tipos de imagens usados foram a partir de dados do sensor GeoEye-1.  O tipo de árvores estudadas foram Citrus. Para a construção do algoritmo, o mesmo possui 4 etapas: Filtro de suavização assimétrico; filtro mínimo local; camada de máscara e operador de agregação espacial. O objetivo deste trabalho foi contar as árvores. Nele descreve as dificuldades obtidas. Os autores não mencionam a porcentagem de acertos, e sim, o que o algoritmo deles precisa melhorar para desempenhar melhor, no auxilio a agricultura de precisão.


%Applying Image Analysis and Probabilistic Techniques for Counting Olive Trees in High-Resolution Satellite Images
%(Aplicação de análise de imagem e técnicas probabilísticas para contagem de oliveiras em imagens de satélite de alta resolução) - González 2007
Segundo o autor \citeonline{Gonzalez2007}, o tipo de imagem usadas são a partir de imagens de satélite de alta resolução que são do sul da Espanha, o tipo de plantas específicas contadas foram Oliveiras.  Os métodos usados para fazer a contagem de Oliveiras, foram pegar as características e o tamanho semelhante das árvores primeiramente, depois produz uma medida probabilística que já detecta qual é uma Oliveira. As técnicas usadas foram a localização dos centróides e usaram escala de cinzas que fornecem resultados melhores.  Usaram a linguagem c++ e biblioteca OpenCV. Por fim, ele mostra a tabela com os resultados de falsos positivos e negativos na contagem dessas Oliveiras.

%DETECTION AND COUNTING OF ORCHARD TREES FROM VHR IMAGES USING A
%GEOMETRICAL-OPTICAL MODEL AND MARKED TEMPLATE MATCHING – MAILLARD 2016        
Segundo \citeonline{Maillard2016}, as imagens estudadas por eles são de alta resolução. As mesmas foram extraídas de diversos lugares do mundo, a partir do aplicativo web Google Earth. O objetivo deles é detectar e contar árvores. Usaram a abordagem correspondência de modelo óptico geométrico(GOM). O algoritmo feito neste trabalho, contém os seguintes estágios: 1. O usuário insere a porcentagem de sobreposição permitida; 2. Se o usuário não inserir a entrada, o próprio programa calcula a semelhança entre a amostra e todas as possibilidades de parâmetros de iluminação; 3. Foi calculado um valor de similaridade para cada pixel; 4. Se a similaridade for multiespectral, repita o estágio 3 para todas as bandas e tornar a similaridade cumulativa; 5.Classifique os pixels diminuindo a semelhança e armazene as coordenadas(O valor tem que ser maior que o mínimo permitido); 6. Coloque uma marca de árvore temporária(círculo) no próximo local de pixel com o maior valor de similaridade; 7. Verifique se o espaço já está ocupado por uma árvore. Se alguma sobreposição for permitida, verifique se o número de pixels diferentes de zero é menor que a porcentagem de sobreposição permitido; 8. Validação dos resultados. Por fim, como resultado das laranjas, nozes e manga tiverem mais de 90\% de precisão, já as macieiras tiveram abaixo de 75\% de precisão. 

%Classification of hazelnut by self-organizing maps
%(Classificação de pomares de avelã por mapas auto-organizados) – Tasdemir 2010
No trabalho de \citeonline{tasdemir2010}, as imagens usadas para fazer a sua pesquisa são a partir de imagens de sensoriamento remoto. O tipo de planta específico que foi estudada é Avelãs. A abordagem que os autores usaram, foi mesclar as informações espectrais e espaciais. Proporam um mapa auto-organizado que pega essas informações sem cálculo adicional de textura. A metodologia que eles usaram é encontrar a cobertura do solo a partir do método baseado em pixels, através de aprendizado de máquina. Primeiro aplicaram a abordagem de aprendizado de quantização de vetores, consideraram cada pixel junto com seus vizinhos em uma janela de tamanho predeterminado e usaram todos os valores espectrais na janela como vetor de característica. Por fim, compararam a abordagem deles com outras usadas em outros trabalhos. 

%Automatic Detection and Segmentation of Orchards Using Very High Resolution Imagery-Aksoy2012
%(Detecção automática e segmentação de pomares usando imagens de alta resolução)
Segundo \citeonline{Aksoy2012}, as imagens usadas neste trabalho foram extraídas no aplicativo web google earth e de satélites de alta resolução. Os autores proporam um algoritmo não-supervisionado para detectar e segmentar espécies das plantas, que são Avelãs e árvores cítricas. Primeiramente, foi feito o aprimoramento de possíveis localizações de árvores usando filtros isotrópicos de granularidade. Posteriormente, a regularidade dos padrões de plantio foi quantificada usando perfis de projeção das respostas do filtro em várias orientações. O resultado foi pontuação de regularidade em cada pixel para cada granularidade e orientação. Já na etapa de segmentação, foi mesclada iterativamente pixels e regiões vizinhos pertencentes a padrões de plantio semelhantes de acordo com as semelhanças de suas pontuações de regularidade e obteve os limites de pomares individuais juntamente com estimativas de suas granularidades e orientações.

%An Automatic Method for Counting Olive Trees in Very High Spatial Remote Sensing Images - Bazi2009
No artigo de \citeonline{Bazi2009} foi usada imagens de sensoriamento remoto de alta resolução espacial. O objetivo dos autores foi efetuar contagem de Oliveiras. O processo que os autores fizeram para fazer a contagem, foi primeiramente aplicar morfologia matemática, em seguida separaram as classes do solo e copas através do classificador gaussiano(HPC), em seguida, o blob que representa as Oliveiras, foram contados de forma automática. Como resultado,  o algoritmo contou 1124 de de 1167 Oliveiras. Segundo os autores, foi um resultado promissor.

%An Automatic Approach for Palm Tree Counting in UAV Images - Bazi2014
Segundo \citeonline{Bazi2014}, o objetivo deles foi fazer contagem de Palmeiras em imagens UAV. Primeiramente, foi feito o treinamento com o SIFT(Scale Invariant Feature Transform) para extrair conjuntos de pontos-chave. Posteriormente, esses pontos-chaves foram analisados com o classificador Extreme Learning Machine (ELM), que também foi treinado no conjunto de pontos-chave de palma e sem palma. Como saída, o classificador ELM marcou cada palmeira detectada por vários pontos-chave. Em seguida, para capturar a forma de cada árvore, foi proposto mesclar esses pontos-chave com um método de contorno ativo baseado em conjuntos de níveis(LS). Por fim, a textura das regiões obtidas por LS com padrões binários locais(LBPs) foi analisada, para distinguir palmeiras de outras vegetações.

%Oil Palm Counting and Age Estimation from WorldView-3 Imagery and LiDAR Data Using an Integrated OBIA Height Model and Regression Analysis - Rizeei2018

No trabalho de \citeonline{Rizeei2018}, foram estudadas imagens a partir do satélite Worldview-3 e de imagens aéreas de detecção e alcance da luz(LiDAR) no ar. Os autores estimaram a idade e contaram Dedenzeiros. Primeiramente, aplicaram o algoritmo de máquina de vetor de suporte(SVM) de análise de imagem baseada em objeto para efetuar a contagem de dendezeiros. Foi feita análise de sensibilidade em quatro tipos de kernel SVM com parâmetros de segmentação associados para obter o melhor delineamento da cobertura da coroa. A extração da copa da árvore foi integrado ao modelo de altura e métodos de multiregressão para estimar com precisão a idade das árvores. O modelo de multiregressão com tamanhos de vários núcleos foi examinado para obter o modelo mais otimizado para estimativa de idade. Os modelos aplicados foram treinados e examinados em cinco diferentes plantações de dendezeiros. Como resultado da contagem, teve 98.80\% de acurácia, já para a estimativa de idade teve 84.91\% de precisão. 

%An yield estimation in citrus orchards via fruit detection and counting using image processing - Dorj2017
Segundo \citeonline{Dorj2017}, as imagens estudadas por eles foram extraídas através de câmera fotográfica. O foco deste trabalho foi detectar e contar citros na árvore.O algoritmo de contagem de citros consistiu nas seguintes etapas: 1. Converter imagem RGB em HSV; 2. Fazer limiarização; 3. Fazer detecção da cor laranja; 4. Fazer remoção de ruído; 5. Fazer segmentação de bacias hidrográficas; 6.Fazer contagem. O algoritmo obteve 93\% de acurácia, e foi comparado com a contagem manual.

%Tree Crown Detection on Multispectral VHR Satellite Imagery - Daliakopoulos2009
No trabalho de \citeonline{Daliakopoulos2009},foram estudadas imagens de satélite multiespectrais de resolução muito alta(VHR). O foco dele foi detectar árvores por meio do seu tamanho, ao invés de espécies específicas.  O método usado, possui combinação dos limites da faixa Vermelha e do Índice de Vegetação de Diferenças Normalizadas(NDVI) e o método de detecção de blob do Laplaciano do Gaussiano(LOG). 

%Utilização de imagens de sensoriamento remoto de alta resolução para realizar a contagem de copas em povoamento de Eucalyptus spp.
No trabalho de \citeonline{Reis2007}, foram usadas imagens de sensoriamento de alta resolução para fazer contagem de Eucalyptus ssp. Primeiramente, foi aplicado o filtro de Lee e em seguida o classificador não-supervisionado ISODATA e foi feita a contagem com o sistema SIG(Sistemas de Informações Geográficas). A abordagem usada por eles obteve 93,58\% de acurácia.

%A graph-based segmentation algorithm for tree crown extraction using airbone LIDAR data – STRIMBU 2015
No trabalho de \citeonline{strimbu2015}, as imagens estudadas foram extraídas a partir do LIDAR aéreo. A floresta que  tos autores estudaram é predominado por pinheiros com as seguintes espécies:  Pinus taeda L. (PT), Pinus palustris Mill. (PP), Quercus falcata Michx. (QF), Quercus alba L. (QA)  Liquidambar
styraciflua L. (LS).  As técnicas aplicadas neste trabalho são as seguintes:%Não sei detalhar as técnicas, vou jogar este artigo no lixo.


\chapter{Metodologia e resultados iniciais}
\label{chapter:desenvolvimento}
\textcolor{red}{Lembrete:TROCAR O TERMO PLANTA POR ÁRVORE NESTE CAPÍTULO}
\section{Considerações iniciais}
Este capítulo descreve a metodologia e resultados iniciais elaborado para fazer a Contagem de árvores usando métodos de Segmentação Watershed e Filtragem máxima local. Foi detalhado os passos do processamento das imagens, como: Desenvolvimento da base de dados de imagens, o planejamento da metodologia, desenvolvimento inicial e resultados iniciais obtidos.
\section{Base de dados de imagens}

Foram coletas imagens de plantações de jabuticabeiras e de coqueiros, utilizando a API Python do Google Maps. A API do Google Maps permite obter imagens de satélite do Google Maps, informando como entrada as coordenadas de latitude e de longitude, e um parâmetro de \textit{zoom}. A API também possui constantes internas que permite definir a dimensão (linhas \versus colunas) da imagem desejada a partir do ponto de latitude e longitude especificado. A API contém um nível de zoo que varia no intervalo de 0 -- 23, sendo 23 o maior nível de \textit{zoom}. Coletamos imagens com nível de \texit{zoom} 18. Todas as imagens de satélite obtidas até o momento foram captadas pelo satélite Airbus/CNES em 2019. Até então nossa base de dados é formada por imagens das seguintes localidades/plantações:

\begin{description}
\item[Fazenda Jabuticabal / plantação de jabuticabeiras]: A Fazenda Jaboticabal é localizada na GO-319, Hidrolândia -- GO, latitude: -16.911523, longitude: -49.360602 (16°54'41.5"Sul, 49°21'38.2" Oeste).

\item[Fazenda em Porto Seguro / plantação de coqueiros]: Fazenda de plantação de coqueiros na Bahia , município de Porto Seguro, latitude: -16.378640, longitude: -39.089385 (16°22'43.1" Sul, 39°05'21.8"" Oeste).
\end{description}

\sergio{No decorrer do desenvolvimento do PFC2 pretende-se coletar um maior número de imagens destas plantações e analisar outras plantação de árvores de outras espécie para a captura de imagens. Além disso pretende-se criar imagens de \textit{ground-truth} para que se possa mensurar quantitativamente a qualidade dos algoritmos de detecção de árvores que serão desenvolvidos.}  

\section{Metodologia planejada}

\sergio{Descrevemos a metodologia planejada em termos dos passos que pretense-se seguir para a identificação e contagem de árvores. Basicamente estes passos são: detecção de vegetação, processamento morfológico, segmentação/detecção de árvores, e rotulação.}

\begin{enumerate}
    \item Detecção de vegetação: basicamente consiste da distinção entre os pixels de vegetação e de não-vegetação (solo, edificações, leitos d'água, estradas, entre outros). Para isto é tradicional na literatura o emprego de índices de vegetação \sergio{que, com base no valores espectrais de pixels, diferencia os pixels de vegetação dos pixels de não-vegetação. Um índice de vegetação normalmente produz um valor para cada pixel, onde valores mais altos indicam uma maior probabilidade deste ser um pixel de vegetação. Como o objetivo dessa etapa é distinguir pixels de vegetação de pixels de não-vegetação, a imagem/mapa de pixels produzida pelo índice de vegetação é binarizada. Existem vários métodos de binarização, sendo o mais tradicionais o método de Otsu~\cite{Gonzales2010} e métodos de binarização local, tal como Sauvola~\cite{Sauvola2000}. Neste trabalho pretendemos explorar os índices de vegetação descritos no Capítulo~\ref{chapter:conceitosPDI} em combinação com o método de binarização de Otsu.}    
    
    \item Processamento morfológico: após a binarização é comum a existência regiões desconexas de uma copa de árvore, copa de árvores distintas conectadas, e de pontos e objetos detectados que não correspondem os objetos de interesse, tal como manchas de ervas daninhas. Para amenizar esses problemas, é comum o emprego de técnicas que fazem uma análise e correção de forma, denominadas de técnicas de morfologia matemática. Estas servem, entre outras coisas, para conexão de objetos, separação de objetos conectados, e eliminação de pequenas regiões. \sergio{Vale observar que alguns problemas que podem ocorrer em imagens, tal como objetos distintos conectados e o mesmo objeto fragmentado, são antagônicos. Assim, deve se investigar e desenvolver uma técnica que melhor atenda a aplicação.} 
    
    \item Segmentação/detecção de árvores: nessa etapa serão aplicados métodos para a detecção de árvores, onde experimentaremos o método baseada no conceito de inundação denominado \textit{watershed} e o método de filtragem máxima local. Para a aplicação do método de filtragem máxima local, aplicamos antes o método de transformada da distância que produzirá valores mais altos no centro das regiões candidatas a copas de árvores.
    
    \item Rotulação/contagem de árvores: Sobre o resultado da segmentação, os da filtragem máxima local aplicam um método de rotulação de componentes conectados para o caso de segmentação, e um simples calculo de contagem dos valores máximos locais, no caso da aplicação do método de filtragem máxima local.
    
    \item Análise dos resultados: será desenvolvido manualmente uma base de imagens de \textit{ground-truth} com a delimitação da copa de cada árvore. Com base nas imagens de \textit{ground-truth} serão mensurados os desempenhos de detecção e contagem de árvores das metodologias desenvolvidas, usando medidas tradicionais da literatura de detecção de objetos, tais como precisão, sensitividade, especificidade, entre outras.  
    
\end{enumerate}

\section{Desenvolvimento inicial}

Nesta seção descrevemos as duas metodologias usadas para a contagem de plantas. Conforme mencionado anteriormente as metodologias compartilham de uma fase inicial de processamento comum e se diferem no método de identificação de árvores. Após a identificação as árvores são rotuladas. A Figura~\ref{fig:metodologia} ilustra as metodologias desenvolvidas e comparadas. Inicialmente é feito uma conversão para níveis de cinza. Em seguida é aplicado o método de Otsu para binarização, com o intuito de separar pixels de plantas de outras classes de pixels. Com o objetivo de preencher buracos nas copas das plantas e eliminar fendas estreitas entre plantas é aplicado um processamento morfológico. Posteriormente é aplicada a transformada da distância com o intuito de dar valores mais altos no centro da copa das plantas e valores mais baixos à medida que se distância do centro da copa da árvore; este é um processamento padrão em vários algoritmos de identificação de objetos. O processamento até este momento é comum à ambas as metodologias. Tomando como entrada a imagem produzida pela transformada da distância, foi aplicado dois métodos para a detecção de plantas: segmentação \textit{watershed} e detecção de picos pela filtragem máxima local. Em seguida as árvores detectadas são rotuladas, sendo o número de rótulos atribuídos igual ao número estimado de árvores.  

\begin{figure}[ht]
\centering
\caption{Passos das metodologias propostas para a contagem de árvores.}
\includegraphics[width=1\textwidth]{images/metodologia_pat.pdf}
\label{fig:metodologia}
Fonte: Elaborada pela autora
\end{figure}


\subsection{Conversão para níveis de cinza}

Para a conversão das imagens para níveis de cinza usamos a medida de luminância~\CITE{Gonzalez2006} que a projetada para casar a percepção humana de brilho, devido a proporção de células sensíveis a cada uma das cores primárias RGB. Assim, é usada combinação ponderada dos canais RGB:

\begin{equation}
\text{Luminância} = 0.2989*R + 0.5870 + 0.1140*B
\end{equation} 

Este cálculo de luminância é o padrão adotado por boa parte das bibliotecas de processamento de imagens para a conversar de imagens coloridas para imagens em níveis de cinza.


\subsection{Binarização}

Após as imagens serem convertidas do sistema de cores RGB  para níveis de cinza, elas foram binarizadas pelo método de Otsu. A ideia do método de Otsu é encontrar o limiar que minimiza a variância ponderada dentro das classes. Ou seja, separa os pixels em dois grupos cuja variância de nível de cinza é a menor possível. Basicamente isto pode ser feito por encontrar o limiar $t$ que minimiza a Equação~\ref{eq:otsu}, onde $p_1(t)$ é a probabilidade de um pixel menor ou igual a $t$, $p_2(t)$ é a probabilidade de um pixel maior que $t$, $\sigma_1^2(t)$ é a variância da classe de pixels menor ou igual a $t$, e $\sigma_2^2(t)$ é a variância da classe de pixels maior que $t$.

\begin{equation}
\sigma_2(t) = p_1(t)\sigma_1^2(t) + p_2(t)\sigma_2^2(t)
\label{eq:otsu}
\end{equation}

\subsection{Tratamento morfológico}
Devido a presença de ruídos que não correspondem a plantas na imagem binarizada, aplicamos duas operações para tratamento morfológico. Primeiro foi aplicado uma operação de preenchecimento de buracos e em seguida foi aplicada uma operação de abertura. 

A abertura, denotada por $A \circ B$, consiste de uma operação de erosão seguida de uma operação de dilatação, usando o mesmo elemento estruturante $B$.  Esta operação elimina pequenos objetos indesejáveis ou pontes estreitas entre objetos, sem que os objetos não eliminados mudem de tamanho radicalmente. Nos experimentos usamos como elemento estruturante um círculo de raio 2.

\subsection{Transformada da distância}

A transformada da distância~\cite{Fabbri2008} mede a distância de cada ponto do objeto até a borda mais próxima. Seja uma imagem bidimensional $I$ consistindo de duas classes de pixels: pixels de objeto (de valor 1) e pixels de não objeto (de valor 0).  Basicamente a transformada de distância atribui a cada pixel de objeto de uma imagem binária, a distância deste pixel para o pixel de borda mais próximo. Matematicamente:

\begin{equation}
I_d(x,y)=
\begin{cases}
0, & \text{se } I (x,y)=0\\
min(||x-x_0, y-y_0||, \forall I(x_0, y_0)=0), & \text{se } I(x,y)=0
\end{cases}
\end{equation} 



\subsection{Segmentação Watershed}

O termo \textit{watershed} se refere à linha de divisória entre bacias hidrográficas. Uma bacia hidrográfica é a área geográfica que drena para um rio ou reservatório específico. Assim transformação da bacia hidrográfica requer que você pense em uma imagem como uma superfície. Você deve imaginar que as áreas claras são altas e as áreas escuras são baixas. Com superfícies, é natural pensar em termos de bacias de captação e linha divisórias entre estas. O algoritmo de segmentação \textit{watershed}~\cite{Gonzalez2006} parte do cálculo de cada mínimo regional de imagem. Em seguida a imagem (superfície) é inundada de baixo para cima gradualmente. Quando a elevação da água em bacias de captação distintas está prestes a se fundir, uma barragem é construída para evitar a fusão. O processo de inundação gradual das bacias de captação é feito por meio de dilatação morfológica. Quando as coordenadas de bacias de captação distintas de encontram é construído uma barreira, normalmente por setar um mais alto que o nível de cinza máximo da imagem, nas coordenas em questão.

\subsection{Filtragem máxima local}

A ideia por trás dos filtros máximos locais~\cite{WULDER2000} para a detecção de copas das árvores é que as copas das árvores, estando mais próximas da fonte de iluminação tem maior refletância. Sob essa suposição, identificar uma copa de árvore se traduz em encontrar, geralmente através de uma janela deslizante, os máximos locais na imagem. Uma das principais questões é que as detecções são muito afetadas pelo tamanho da janela deslizante. Uma janela que é muito grande em comparação com a copa de uma árvore falha ao detectar diferentes copas das árvores, resultando na mesclagem de copas diferentes. Por outro lado, uma janela muito pequena criará muitos falsos positivos, identificando vários pixels brilhantes que pertencem à mesma coroa. Uma seleção cuidadosa do tamanho da janela deslizante é, portanto, fundamental. As coroas das árvores estão longe da forma geométrica precisa. Isso significa que, muitas vezes, devido à topologia das árvores, pixels de alta intensidade podem ocorrer fora da parte mais alta da coroa. Como neste trabalho aplicamos a transformada da distância sob a imagem binarizada, o problema de múltiplos máximos locais em uma mesma copa de árvore tende a ser reduzido.

\subsection{Rotulação}

Por fim é feito uma rotulação de componente conectados, onde é atribuído um rótulo distinto para cada região 8-conectada.  Seja $p$ e $q$ pertencente a objeto $S$. Então $p$ é conectado à $q$ se existe um caminho de $p$ para $q$ consistindo inteiramente de pixels pertencentes à $S$. Para qualquer $p \in S$, o conjunto de pixels em $S$ que são conectados à $p$ é chamado de um componente conectado de $S$. Um caminho (\textit{path}) do pixel $p$ em $(x,y)$ para o pixel $q$ em $(s,t)$ é uma sequência de pixels distintos: $(x_0,y_0), (x_1,y_1), (x_2,y_2),\ldots, (x_n,y_n)$
tal que	$(x_0,y_0) = (x,y)$ e $(x_n,y_n) = (s,t)$
e $(x_i,y_i)$ é adjacente à $(x_{i-1},y_{i-1})$,      $i = 1, \ldots,n$. Neste trabalho consideramos a adjacência de 8, que considere que todos os pixels vizinhos na vertical, horizontal e diagonais são adjacentes. 
   

\section{Resultados}
\label{sec:results}

Os imagens de satélite usadas, baixas através da API do Google Maps são mostradas na Figuras\ref{fig:coco} e \ref{fig:jabuticaba} onde a primeira se refere a uma plantação de cocos e segunda a uma plantação de jabuticabas, obtidas conforme descrito na Seção~\ref{sec:matmet}. Após a obtenção das imagens de satélite, separamos manualmente o talhão refente ao plantio em questão e contamos visualmente a quantidade de plantas. Em seguida aplicamos as duas metodologias às imagens e mensuramos a quantidade de plantas detectadas e a taxa de erro na detecção. A Tabela~\ref{tab:results} mostra o número de plantas para cada imagem, o número de plantas contadas por cada metodologia de a taxa de erro na contagem de plantas. Pode-se observar que a metologia que usa a filtragem máxima local é mais precisa com 92.03 de precisão para imagens de jabuticabeiras, versus 83.94 de precisão da abordagem baseada em \textit{watershed}. Para a contagem de coqueiros a metologia baseada em filtragem máxima local foi mais precisa com 92.88\% de precisão versus 87.91\% de \textit{watershed}. Apesar da filtragem máxima local prover resultados de contagem de plantas mais precisos, este método não delimita a copa das árvores, o que seria inapropriado caso o usuário queira estimar valores como taxa de cobertura do solo, uniformidade das plantas, etc. Por outro lado, \textit{watershed} provê a segmentação das copas das árvores, porém a precisão na contagem foi significativamente inferior a da filtragem máxima local.     

\begin{figure}[ht]
\centering
\caption{Imagem de satélite de uma plantação de coqueiros na Bahia.}
\includegraphics[width=1 \textwidth]{images/coco.png}
\label{fig:coco}

Fonte: Elaborada pela autora.
\end{figure}


\begin{figure}[ht]
\centering
\caption{Imagem de satélite de uma plantação de jabuticabeiras em Hidrolândia, Goiás.}
\includegraphics[width=0.98\textwidth]{images/jabuticaba.png}
\label{fig:jabuticaba}

Fonte: Elaborada pela autora.
\end{figure}

\begin{table}[h]
\caption{Resultados de contagem de árvores de jabuticabeira e coqueiros manual e dadas metodologias desenvolvidas.}

\begin{center}
\begin{tabular}{ |l|l|l|l|l|l| } 
\hline
\multirow{2}{4em}{Espécie} & \multicolumn{3}{|c|}{Quantidade de árvores} & \multicolumn{2}{|c|}{Taxa de acerto}\\
 & manual & LM & watershed & LM & watershed\\ 
\hline
Jabuticabeiras	& 1569	& 1444	& 1821	& 92.03\%	& 83.94\%\\
\hline
coqueiros &	1489 & 1383	& 1669	& 92.88\% &	87.91\%\\
\hline
\end{tabular}
\end{center}
\centering{ Fonte:Autoria própria.}
\label{tab:results}
\end{table}

\section{Considerações finais}
Este capítulo apresentou o desenvolvimento inicial deste trabalho para fazer a contagem de árvores usando métodos de Segmentação Watershed e Filtragem máxima local. Foi abordado os passos do processamento das imagens, como: desenvolvimento da base de dados de imagens, o planejamento da metodologia, desenvolvimento inicial e resultados iniciais obtidos.

%\chapter{Metodologia e Cronograma}
%\label{chapter:metodologia}
%\input{tex/metodologia}

\chapter{Cronograma}
\label{chapter:cronograma}
O cronograma do projeto, com duração de 12 meses, é dado pela Tabela~\ref{cronograma}, onde as atividades marcadas com V foram concluídas e as atividades com X são as tarefas futuras. O cronograma apresenta as seguintes atividades:

\textbf{1. Revisão bibliográfica:} Levantamento e estudo de artigos relacionados aos problemas do projeto.

\textbf{2. Construção da base de dados:} Escolha de espécies de árvores e plantações para captura de imagens, coleta de imagens e desenvolvimento de imagens de ground-truth contendo a delimitação de copa das árvores.

\textbf{3. Desenvolvimento da metodologia e implementação:} Desenvolvimento da metodologia para detecção e contagem de árvores baseado no método de segmentação por inundação (watershed) e baseada na detecção de picos, implementação e ajuste de parâmetros das técnicas.

\textbf{4. Escrita de artigos:} Escrita de artigos relacionados ao projeto para publicações em eventos.

\textbf{5. Defesa de PFC-1:} Apresentação da monografia para uma banca avaliadora, na disciplina de PFC-1.

\textbf{6. Elaboração e execução dos testes:} Fase de experimentos da implementação já finalizada, seguida por refinamentos.

\textbf{7. Validação:} Análise e conclusão dos resultados obtidos através das variadas medidas de precisão.

\textbf{8. Defesa de PFC-2:} Apresentação final da monografia para uma banca avaliadora, na disciplina de PFC-2.

% tabela com 25 colunas e 15 linhas, caption de tabela vem acima da mesma.
\begin{center}
\begin{table}[h]
\caption{Cronograma de atividades} % mude aqui para seu título da tabela

\begin{tabular}{|p{5cm}|l|l|l|l|l|l|l|l|l|l|l|l|}
\hline
\multicolumn{1}{|c|}{\textbf{Atividades}}\multirow & \multicolumn{12}{c|}{\textbf{Cronograma (em meses)}}\\  \cline{2-25} \hline
\multicolumn{1}{|c|}{} & 01 & 02 & 03 & 04 & 05 & 06 & 07 & 08 & 09 & 10 & 11 & 12 \\ \hline
Revisão bibliográfica & ~ & ~ & V~ & V~ & V~ & V~ & ~ & X~ & ~ & X~ & ~ & X~  \\ \hline
Construção da base de dados & ~ & ~ &V ~ & V~ & V~ & V~ & ~ & X~ & X~ & ~ & ~ & ~  \\ \hline
Desenvolvimento da metodologia e implementação & ~ & ~ & ~ & V~ & V~ & V~ & ~ & X~ & X~ & X~ & X~ & ~  \\ \hline
Escrita de artigos & ~ & ~ & ~ & V~ & V~ & V~ & ~ & X~ & X~ & X~ & X~ & ~  \\ \hline
Defesa de PFC-1 & ~ & ~ & ~ & ~ & ~ & X~ & ~ & ~ & ~ & ~ & ~ & ~  \\ \hline
Elaboração e execução dos testes & ~ & ~ & ~ & ~ & ~ & ~ & ~ & ~ & ~ & X~ & X~ & ~  \\ \hline
Validação  & ~ & ~ & ~ & ~ & ~ & ~ & ~ & ~ & ~ & ~ & X~ & ~  \\ \hline
Defesa de PFC-2 & ~ & ~ & ~ & ~ & ~ & ~ & ~ & ~ & ~ & ~ & ~ & X~  \\ \hline

\end{tabular}
\label{cronograma} 
\centering{Fonte: Elaborada pela autora}
\end{table}
\end{center}

\chapter{Conclusão}
\label{chapter:conclusao}
Contagem de plantas é importante para várias análises agrícolas como estimativa de produtividade, verificação de mortalidade de plantas que pode estar relacionada a doenças ou condições meteorológicas, e inventários que pode servir para várias estimativas como de massa de carbono, informações para planejamento de irrigação, entre outras.\cite{Daliakopoulos2009} 

Nesta pesquisa foram propostas duas metodologias para a contagem de plantas, sendo que elas têm uma fase de processamento comum (conversão das imagens para níveis de cinza, binarização, tratamento morfológico, e aplicação da transformada da distância). Em seguida são experimentados dois métodos para a detecção das plantas (segmentação \textit{watershed} e detecção de picos através da filtragem máxima local). Posteriormente é feita uma rotulação da plantas identificadas. 

Ao aplicar as metodologias para imagens de satélite coletadas pela API Google Maps, o método que usa a filtragem máxima local obtém resultados mais precisos, com 92.03\% de precisão para a contagem de jabuticabeiras e 92.88\% para a contagem de coqueiros, versus 83.94\% e 87.91\%, respectivamente, obtidos pela metodologia que usa segmentação \textit{watershed}. 

É importante destacar, que esta forma de análise dos resultados é propensa a erros, pois uma estimativa satisfatória do número de plantas não significa necessariamente que as plantas foram detectadas corretamente. Assim, na disciplina de PFC-2 será aprimorada aplicando métodos de avaliação da detecção de plantas mais robustos, que se baseia em critério como de falsos positivos, falsos negativos e verdadeiros positivos. Contudo, para a aplicação dessas medidas é necessário identificar manualmente cada planta na imagem, que será etapa feita futuramente. Além disso, será feito experimentos de índices de vegetação para a detecção dos pixels de vegetação, e usar informações geométricas, como o espaçamento entre plantas para eliminar falsos positivos.

% ---
% Finaliza a parte no bookmark do PDF, para que se inicie o bookmark na raiz
% ---
\bookmarksetup{startatroot}% 
% ---

% ----------------------------------------------------------
% ELEMENTOS PÓS-TEXTUAIS
% ----------------------------------------------------------
\postextual

% ----------------------------------------------------------
% Referências bibliográficas
% ----------------------------------------------------------
\bibliography{references}

% ---------------------------------------------------------------------
% GLOSSÁRIO
% ---------------------------------------------------------------------

% Arquivo que contém as definições que vão aparecer no glossário
\input{tex/glossario}
% Comando para incluir todas as definições do arquivo glossario.tex
\glsaddall
% Impressão do glossário
\printglossaries

% ----------------------------------------------------------
% Apêndices
% ----------------------------------------------------------

% ---
% Inicia os apêndices
% ---

% ---


% ----------------------------------------------------------
% Anexos
% ----------------------------------------------------------

% ---
% Inicia os anexos
% ---

\end{document}